Transcript of PsXXYborg:

Rapture of the Nerds. 
I haven't had any internet at home for about a week. I figured that out today. This is what the van looked like, and we were all super proud of the van, which has gone to its final resting place. 

So... all of mine is in text. I hope you like reading.... because what I do is text. I was not heavily involved in any of the theory parts of this project, or much of the artistic development parts of the process. I heard Feminist Art Game and Donna Haraway and I thought...MMM! Excellent! These are things I have studied! I have written a lot about these things! I can probably do this. And when I was invited to the project, we didn't really have a person to make the dual-screen part ... happen. And then we tried to hire a couple of them, and that did not go well. So the net result was that I would go over to Hannah's place, and I would hang out, and they would say "How are we going to DO this?" and I would say "aaaaaaauuhhhm, I've heard of this thing! Somebody told me about it last week, we'll use that." 

This is not really an easy or sensible thing to do. Most people come to coding [a new project] already knowing a language, and they're comfortable with that language, and that language has conventions and assumptions, and ways it works. The language that psXXYborg is written in is Javascript, which until recently was not available for use on the server. It was only available for use in the browser. I'm sorry if none of this makes any sense to you. You can ask me later. 

[Learning Javascript and writing psXXYborg in it] is exciting and very very new, and it was on the basis of a lot of money. Google has paid a lot of money for the technology that drives psXXYborg to exist, and it really wants us to use it. So I had heard of Node, but I'd never really touched it, and I hadn't written more than about four lines of javascript in my life before we started this project. But I said "oh, we'll use Node and that will be fine." I think maybe Cecily and Jennie understand how much of a joke learning an entirely new computer language for a single project is. 

psXXYborg is about new ideas, so it runs using new ideas. These are ideas about the direction of the web, and how to use it. Google has very specific ideas about how it would like the web used, and we used some of those ideas, which are about interactivity and the Chrome browser, to make this thing work. It works in ways that it shouldn't work. It works using a programming language that was not designed to build streaming web servers, and it uses the very newest ideas to do that, so a lot of psXXYborg was about research.

Coding in isolation is difficult, because code is about making ideas into actions. Code is a language that is about assumptions, and it's about driving machines to do what you want them to do, to make your own world. As a technologist, coming up with new ideas that are actually useful to people is almost impossible. It's just very, very difficult. People will tell you that is not true, that they have wonderful ideas: the problem is that everybody else has pretty much the same ideas, because tech builds on tech, the way that lanugage builds on language. English picks up bits and pieces of other words from other languages like a thief and technology advances upon itself to develop new things. So coming up with something that's genuinely new using a technological driver is really difficult, because the people who are already thinking about technology are not thinking about technology for what it might make [the humane], they're thinking about making more technology [more stuff]. 

Without a project goal, there's nothing to write. If you don't have a thing to make, you're not making a thing. You're arranging some ideas on a page, and that's all. And in isolation, that gets very lonely, because you do need to concentrate when you're coding, you need to concentrate for gigantic blocks of time - it's a creative practice that way. It's very hard to hold ideas in your head for a long period of time if you have to go off to meetings and do other things. So this is a creative practice that's about isolation that cannot be driven alone. HARD.

psXXYborg offered me an excellent opportunity to think about how new, expensive, "free" technologies might be used to make hard things easier. Technically hard things. The magic in code is that it disappears. You only see code when something goes wrong, because otherwise, it simply acts as though it is a glass panel. And that is mostly what technologists want code to do. We pretty much want it to retreat into the background, because when it's centre stage and when it's on display, people become frightened of it, sort of like how if you put a word problem in front of a 13-year-old, they screetch and run to the other room. Just a problem. It's a thing that happens. So until something goes wrong, code is invisible.

The joy of fixing things is that when they go right, the code disappears again. So that's the whole idea about the invisibility of code. I'm working on that further for my thesis, because I think it's a big idea, and it's a good one to write down in the context of the arts.

There are lots of interesting threads in the disappearance of a language that drives the mechanics of your world. These are languages that build your medical devices, your cars, your phone: having them be invisible is an interesting problem the same way the underlying gendered assumptions in language is an interesting problem to overcome from the point of view of the legal system. It's an interesting problem. What I mean by an interesting problem is a problem that's worth spending a long time on, teasing out all the details. It's like a math problem that's made up of words. Not a word problem, a problem with the words. So, language assumes culture, Javascript assumes callbacks. That means you write it in reverse. It's sort of like being Ginger, you dance backwards in heels the whole time.

As a coder, what you write becomes real in a tangible sense, making interesting new things is difficult, thinking about what they might be is hardest, particularly alone - so, how do you fix this? You find a gang. A girl gang! The arts have an audience, but their audience is limited by technology [\cite{LISA notes}. If you have a particularly brilliant painter or a particularly brilliant graphic designer or an awesomely amazing musician, that is very fine, however, if they cannot record their information, or transmit it to a broader audience, then they will be broke. We know this is a problem. The next problem is how can we line up the technology so that they don't go broke after they do have the audience. We're still working on that one. It's a scary one. 

Technology has a profit margin, but technology, because it builds on itself in a recursive manner, after education and privilege has given people the ability to write it, often has a very limited cultural outlook. People burn out in the tech field all the time. They burn out constantly. Google hires people for the express purpose of burning them out building machines. Because that's what we do. We write documents that are machines. They're laws. They're just laws made of a different kind of language. And they are breakable. There's an entire body of people who work breaking the laws of the language that we write into our documents. We call them hackers. They're very valuable. We need people who can break the laws to show us where they go wrong. 

Together, good art and good technology can make good experiences. What I mean by good experiences is experiences that ask questions or display well-rounded characters or use any of the many guidelines and theory that we have that drive the Humanities and the arts more broadly. If we pair people off, then we lose the part where the technologists are scared of the artists because the artists speak a mysterious language that the technologists do not understand, and the artists can learn that what looks like magic - a beautiful glass panel that just does what they say - actually takes a lot of work, it's not work that is unquantifiable either. It's built in code commits and checkins, you can see where the period goes, it isn't being a wizard. On either side. It's being a mechanic. It's assembling things well and putting them together and bolting them down and recording that so that it can be done again by another person. This is true on both sides of the equation. Feminism is equality. Art is technology. 

By driving with art, rather than technology, the experience can be newer, and less expected. This is because of the aforementioned recursiveness of technology. Tech builds on tech, almost unquestioned, but art builds on culture and culture has a way of slipping around. When you are not looking at it, it moves in the dark. You think that teddy bear is sitting on your shelf but you wake up and it's at the end of your bed and let me tell you, you will scream. Culture can be seen as nothing but questions, questions like "How might a person be?" I think the answer is better. 