% Chapter Template

\chapter{Central Thesis Argument} % Main chapter title

\label{Chapter8} % Change X to a consecutive number; for referencing this chapter elsewhere, use \ref{ChapterX}

\lhead{Chapter 8. \emph{Central Thesis Argument}} % Change X to a consecutive number; this is for the header on each page - perhaps a shortened title

%----------------------------------------------------------------------------------------
%	Artists Need Accessible Technology
%----------------------------------------------------------------------------------------

\section{Artists Need Accessible Technology}

Technology is no good to anyone if it remains only for use by those who have enough money to buy the newest shiniest things every year, because that means that we will remain stuck with art made out of the parts that the main body of society does not connect with. It also means that art will remain largely material. Art which is immaterial is difficult to collect and to store, even more difficult than contemporary art fabricated out of - for example - plush toys, as the work of Mike Kelley in L.A.

Artists have a long history of hacking together work out of whatever comes to hand. This is useful for resistance, but less good for strained gallery systems. Even a few pieces that take advantage of the power of contemporary - or affordably almost-contemporary - systems can make their metier more accessible, and with that accessibility, more able to locate a diverse audience. 


%-----------------------------------
%	Code is a form of creative language. A program is a type of essay.
%-----------------------------------
\subsection{Code as a creative form of language}

This is an argument about the design of writing, and following on that, what writing designs. Code is writing that designs the way that a machine should work, far more directly than any previous form of written language. Where a blueprint allows for human error, code is made of zeroes and ones: it fires properly, or it does not fire properly, at least to begin with. Later, there is space for argument. If there is space for argument, people will argue over what is best, or right, or wrong, and this is how you can tell that code is a creative practice.

 Writing is a difficult trade, in one language or another. Writing, straight writing, is the assembly of language to represent ideas. In essay format, those ideas include libraries: the writing done by other people, in one country or another, to bolster an argument. Code is like this writing, in that it is interdependent: it depends on external libraries of information in order to function. When we include these libraries, we preserve their licencing, which frequently includes their authorship. The joy of fixing things is that when they go right, the code disappears again. 

This practice is very similar to the practice of academics or lawyers, structuring new ideas and new laws, respectively, on the backs of one another. The difference is that the code we write can directly control machines, which in turn provide the framework for other activities. It is also somewhat more difficult to trick code into breaking itself than it is to break a law.

Being a creative work of structured language, code is reliant on many libraries, themselves code, written by previous programmers with their own expectations. Code is delicate: it depends on hardware systems and software systems both to be consistent and well-articulated in order to function, to do what it has been written to do. In this, code is not so different than an essay, or any highly conceptual creative practice, all of which rely on some degree of education to communicate their themes and ideas.

This complicates coded practice with politics, however. The best new systems were, for a period, likely to be produced by individuals with a superfluity of time and access - capital formats that are associated with luck and conventional constructions of social privilege. This means that many hackers - a word here used in the MIT sense of "constructive re-users of technology" - came from an exclusive position. One might even think all of them were destined to do this. Unfortunately, this is not the case: in the 1970s, programming was advertised to women as a reasonable career even in Cosmopolitan magazine \cite{ensmenger1}. It was not until later that programming came to be associated, through a system of hypercompetitive standardized tests, with masculine traits related to the ideal of math and science as the height of rational - manly - work.

Code is not necessarily manly, or unmanly, or gendered at all, until someone puts a gendered pronoun into their documentation, and thus demonstrates their assumptions about their users. Presenting gender as visible in code is unfortunate: it is an avoidable, simple grammatic change to shift a library from being a nominally more-neutral territory to one that is passively armoured against outsiders, as demonstrated by Miller and James in their paper "Is the generic pronoun he still comprehended as excluding women?" \cite{pronounscience}. In this case, the outsider would be someone who would prefer a neutral pronoun, or a female one: this is the sort of thing that matters a great deal in writing, particularly in writing with heavily gendered languages such as French, but which becomes elided when it is expressed in the apparently neutral world of technical production. 

I would argue, elsewhere, that this is because the technical world has been so coherently reconstructed around competitive performances of masculinity. This argument is too broad for the scope of this particular paper, however: arguments around masculinity and its constructions compose entire faculties. This is intended to be a small paper, about the importance of access and privacy, and how software tools can permit artists to work more freely with more of both. A simple tool can articulate things inarticulable by something more complex. 

In English and in the world of industrial design, which code most certainly is, the best arguments for design simplicity are made in "The Design of Everyday Things" by DA Norman. Norman argues that "far too many items in the world are designed, constructed, and foisted upon us with no understanding - or even care - for how we will use them" \cite{everydaythings}. While Norman is there referring to objects, it is clear from the complexity and complaints around any major software package that there are similar upsets to be found within software. The more difficult it is to set up a new interaction, the less time there is for designing what the content of that interaction should be. Good design also recedes. It is the bad that comes forward. In this case, the bad which comes forward is comprised almost entirely of linguistic assumptions in design: things that disappear as neutral when they should appear as distinct, and problematic.

The easiest writing to absorb on this topic is not apparently about design at all. It is, instead, about desire - but the design of objects is the design of that which is desired, within a capitalist market-driven system, because one must design, and then spend vast amounts of capital to build, that which others might buy. The resonance of writing, design, and desire to poststructural feminism is made stronger when one considers the works of previous feminists. The suffragettes who got ladies the vote also sold them their soap via advertising firms at the turn of the century \cite{consumerbeauty}.This cycle carries forward and forward and forward: women are people, people are people, there is a troubling disparity in which work is assigned to whom when, which carries forward invisibly into an understanding of who is allowed to use which tools to do what work, also known as privilege.

There are lots of interesting threads in the disappearance of a language that drives the mechanics of a world. Computer programming language is the language that builds medical devices, cars, phones. That such language recedes when good - thus remaining unexamined for long stretches of time - and then appears when buggy is an interesting problem, one that is a challenge to overcome. This is where it becomes useful to look outside of computer science to begin to have an idea of how encoded ideals within language can influence the effect one can have one one's world. The best-known and most straightforward theorist on this topic is H�l�ne Cixous, who wrote in 1964 an essay called the Laugh of the Medusa.

Cixous spent a lot of the Laugh of the Medusa on the idea of direct, physical desire. This has some interesting resonance with the popularization of code-communication, because pornography sprouted in the comfortable anonymity of the internet like mushrooms sprout in forests after the rain, so clearly someone's desires were not being met by the extant system of expression. This resonance is outside the scope of my thesis project, although I believe it is interesting to note that the majority of young women, given a dual-screen game, made games about dating, directly about gender, or in the case of the finest producer - a transhuman - pornography itself. Embodiment is clearly very important to these people. The state of the personal narrative in a new medium is entirely too broad for me to address here, however.

Therefore, the part of Cixous that is interesting for understanding the idea of code as a creative practice of resistance is her insistance that women write, and that women write to resist the language in which they are cast. French is a strongly gendered language. It places a gender on each noun, and a degree of familiarity on the use of the term "you" that we cannot yet articulate within English: the best we can do is "they," which is pluralized in a fashion that "vous" is not. This is particularly interesting in Canada, a country of dual languages, where schoolchildren are trained from a young age to have at least a passing fluency with both worlds. We are expected to learn between two countries: it is illegal to exclude the exoticized European other from our signage. 

 Language is a politicized site of visibility within Canada. This forced inclusivity, which itself still elides much Canadian experience, is important to my understanding of code as a practice of linguistic structure. Cixous' particular articulation of the value of the minority writing their own world, even when forced to use a dominant language - that itself is constructed to preclude their existence - is a valueable view of how resistance to invisibility might be realized.

In code, this comes because code languages, and the libraries that make them work, are released by major corporations on a regular basis. Innovations become dependent on an ability to fall in line with the way of thinking that hundreds or thousands of previous workers have made happen, while still preserving a sense of creativity that permits one to figure out basic problems. These problems are largely technical: how to make best use of a second screen, how to force a specific video to play back in that framework. Everything inside a computer is numbers, however, and the way those numbers line up has been dictated by politics. ScreenPerfect, as psXXYborg, was written to take advantage of the Safari browser - backed by Apple - but webM video files are backed by Google, and are more compact. Due to complex negotiations around copyright, the Apple's webkit solution will not use webM, and Chrome would not play back H.264. 

These differences are often put down to simple "technical problems," but this idea is inaccurate, and elides the code that underlies the systems built by large companies to transmit information. Video files are a contested ground at the moment, because in addition to being a convenient communication medium, some video files are incredibly protected forms of information. The litigation surrounding piracy of television and conventional first-run media has been publicly associated with theft: a type of theft where an object's value is removed even while the object itself remains present and resellable. Patent fights about video format, about technical standards, and about open or closed systems, dictate the terms of how media may be distributed and displayed. 

screenPerfect gets around many of the restrictions on video by coding a new system for display and interaction, which values a short, engaged experience over the longer-form systems that are already in place. Rather than pursuing the television experience, SP turns video to a system more in line with video games, specifically riffing on an engine popular with independent game-makers. Independent game-makers are concerned with a new medium, the video game, which is made up of the interaction of people with the game system. As Galloway argues in his essay on gaming, the software alone does not count.

Video games are new, and therefore they are as-yet unformalized: the best we get is a system of triple-A games, which are largely concerned with imposing men shooting things with imposing guns. The triple-A framework is not typically particularly feminist, with one or two startling exceptions - the end of Saint's Row 4, which is a feminist game through and through, features the rescue of Jane Austen by space aliens, for example. By and large, the popular and commercial nature of large games precludes the sort of deeply personal representation that Cixious refers to in her own work. 

Games - code activated by interaction - are not restricted to only the blockbuster, however, any more than film is. Games are sometimes accused of Cinema Envy, and I believe this is a real problem: because they are not yet taken seriously, games are still free to explore new ideas, to be obscene or funny or reflective of their own culture. Jane Austen is peculiarly popularly persistent, after all: even post-Bridget Jones, the Austen canon continues to inspire new works of popular culture. The important part, however, is that game-makers are still relatively free artists. The code they use, based on massive information systems, still has the potential to make money and drive creativity in the use of and development of technology.

Innovation is not a worthwhile term here, because it has recently been devalued to refer primarily to systems that can be developed quickly and deployed to a broad audience. My interest is not in the broad audience, although it is simple good practice to write code that can be widely distributed. My interest is in the support and production of technology that permits new forms of private expression. Here, too, I do not mean private in the sense of overly limited. I mean private in the sense that something may be presented to a limited audience with the understanding that the complete experience is meant to be retained by that audience. At the same time, the technology is intended to take advantage of systems already in broad use, because these are systems accessible to artists, who must work with what is presently real to define what might become real.

In this, it is important that the privacy of a system not be restricted to exclusively those with the major capital to install and control a given band of technology. While developing the dual-screen technology to run psXXYborg, I was considering a scene from Cory Doctorow's Pirate Cinema (2012), in which a crowd of young film-makers, who make their work entirely from pirated media that has been remixed and repurposed, put up a movie theatre in a forested park by synchronizing the pico projectors on their phones. This was the chief inspiration for screenPerfect: a technology so lightweight that it would require no setup and no particular technical skill to use, which could then permit artists who already had clear vision the advantage of being able to screen their works anywhere, quickly, with nothing to lose.

Pico projectors have not yet taken off in popularity, because they have not yet begun to be built into smartphones and cell phones. That does not matter. We have a laptop on every table, and on those laptops are browsers. To write software that can be understood by the browser, one must interact with a system of capital that is dedicated to the co-option and devaluation of the author at the privilege of the corporation (TVO interview, Sawyer). The point, then, as a practitioner of a form of creativity that has not yet been completely co-opted - for the creativity of someone solving a problem within code, a specific problem especially, is still a creative practice - is that we may resist and open a door to further resistance. 



%-----------------------------------
%	Technology relies on itself to be built, so there are political questions inside that building.
%-----------------------------------

\subsection{Political questions of interdependence of code on corporate support}
Morbi rutrum odio eget arcu adipiscing sodales. Aenean et purus a est pulvinar pellentesque. Cras in elit neque, quis varius elit. Phasellus fringilla, nibh eu tempus venenatis, dolor elit posuere quam, quis adipiscing urna leo nec orci. Sed nec nulla auctor odio aliquet consequat. Ut nec nulla in ante ullamcorper aliquam at sed dolor. Phasellus fermentum magna in augue gravida cursus. Cras sed pretium lorem. Pellentesque eget ornare odio. Proin accumsan, massa viverra cursus pharetra, ipsum nisi lobortis velit, a malesuada dolor lorem eu neque.

%----------------------------------------------------------------------------------------
%	SECTION 2
%----------------------------------------------------------------------------------------

\section{Leftover psXXYborg notes}
The joy of fixing things is that when they go right, the code disappears again. So that's the whole idea about the invisibility of code. I'm working on that further for my thesis, because I think it's a big idea, and it's a good one to write down in the context of the arts.

There are lots of interesting threads in the disappearance of a language that drives the mechanics of a world. These are languages that build medical devices, cars,  
phones. The invisibility of these languages is an interesting problem in the same respect that gendered linguistic assumptions are an interesting problem. because they are 
invisible, they are a challenge to overcome. This is where it becomes useful to look outside of computer science to begin to have an idea of how encoded ideals within 
language can influence the effect one can have one one's world. The best-known and most straightforward theorist on this topic is H�l�ne Cixous, who wrote in 1964 an essay 
called the Laugh of the Medusa.

Cixous spent a lot of the Laugh of the Medusa on the idea of direct, physical desire. This has some interesting resonance with the popularization of code-communication, because pornography sprouted in the comfortable anonymity of the internet like mushrooms sprout in forests after the rain, so clearly someone's desires were not being met by the extant system of expression. This resonance is outside the scope of my thesis project, although I believe it is interesting to note that the majority of young women, given a dual-screen game, made games about dating, directly about gender, or in the case of the finest producer - a transhuman - pornography itself. Embodiment is clearly very important to these people. The state of the personal narrative in a new medium is entirely too broad for me to address here, however.

Therefore, the part of Cixous that is interesting for understanding the idea of code as a creative practice of resistance is her insistance that women write, and that women write to resist the language in which they are cast. French is a strongly gendered language. It places a gender on each noun, and a degree of familiarity on the use of the term "you" that we cannot yet articulate within English: the best we can do is "they," which is pluralized in a fashion that "vous" is not. This is particularly interesting in Canada, a country of dual languages, where schoolchildren are trained from a young age to have at least a passing fluency with both worlds. We are expected to learn between, it is illegal to exclude the other from our signage. This is important to my understanding of code as a practice of linguistic structures. Language is a politicized site of visibility within Canada. Cixous' particular articulation of the value of the minority writing their own world, even when forced to use a dominant language that itself is constructed to preclude their existence, is a valueable view of how resistance might be realized.

In code, this comes because code languages, and the libraries that make them work, are released by major corporations on a regular basis. Innovations become dependent on an ability to fall in line with the way of thinking that hundreds or thousands of previous workers have made happen, while still preserving a sense of creativity that permits one to figure out basic problems. These problems are largely technical: how to make best use of a second screen, how to force a specific video to play back in that framework. Everything inside a computer is numbers, however, and the way those numbers line up has been dictated by politics. ScreenPerfect, as psXXYborg, was written to take advantage of the Safari browser - backed by Apple - but webM video files are backed by Google, and are more compact. Due to complex negotiations around copyright, the Apple's webkit solution will not use webM, and Chrome would not play back H.264. 

These differences are often put down to simple "technical problems," but this idea is inaccurate, and elides the code that underlies the systems built by large companies to transmit information. Video files are a contested ground at the moment, because in addition to being a convenient communication medium, some video files are incredibly protected forms of information. The litigation surrounding piracy of television and conventional first-run media has been publicly associated with theft: a type of theft where an object's value is removed even while the object itself remains present and resellable. Patent fights about video format, about technical standards, and about open or closed systems, dictate the terms of how media may be distributed and displayed. 

screenPerfect gets around many of the restrictions on video by coding a new system for display and interaction, which values a short, engaged experience over the longer-form systems that are already in place. Rather than pursuing the television experience, SP turns video to a system more in line with video games, specifically riffing on an engine popular with independent game-makers. Independent game-makers are concerned with a new medium, the video game, which is made up of the interaction of people with the game system. As Galloway argues in his essay on gaming, the software alone does not count.

Video games are new, and therefore they are as-yet unformalized: the best we get is a system of triple-A games, which are largely concerned with imposing men shooting things with imposing guns. The triple-A framework is not typically particularly feminist, with one or two startling exceptions - the end of Saint's Row 4, which is a feminist game through and through, features the rescue of Jane Austen by space aliens, for example. By and large, the popular and commercial nature of large games precludes the sort of deeply personal representation that Cixious refers to in her own work. 

Games - code activated by interaction - are not restricted to only the blockbuster, however, any more than film is. Games are sometimes accused of Cinema Envy, and I believe this is a real problem: because they are not yet taken seriously, games are still free to explore new ideas, to be obscene or funny or reflective of their own culture. Jane Austen is peculiarly popularly persistent, after all: even post-Bridget Jones, the Austen canon continues to inspire new works of popular culture. The important part, however, is that game-makers are still relatively free artists. The code they use, based on massive information systems, still has the potential to make money and drive creativity in the use of and development of technology.

Innovation is not a worthwhile term here, because it has recently been devalued to refer primarily to systems that can be developed quickly and deployed to a broad audience. My interest is not in the broad audience, although it is simple good practice to write code that can be widely distributed. My interest is in the support and production of technology that permits new forms of private expression. Here, too, I do not mean private in the sense of overly limited. I mean private in the sense that something may be presented to a limited audience with the understanding that the complete experience is meant to be retained by that audience. At the same time, the technology is intended to take advantage of systems already in broad use, because these are systems accessible to artists, who must work with what is presently real to define what might become real.

In this, it is important that the privacy of a system not be restricted to exclusively those with the major capital to install and control a given band of technology. While developing the dual-screen technology to run psXXYborg, I was considering a scene from Cory Doctorow's Pirate Cinema (2012), in which a crowd of young film-makers, who make their work entirely from pirated media that has been remixed and repurposed, put up a movie theatre in a forested park by synchronizing the pico projectors on their phones. This was the chief inspiration for screenPerfect: a technology so lightweight that it would require no setup and no particular technical skill to use, which could then permit artists who already had clear vision the advantage of being able to screen their works anywhere, quickly, with nothing to lose.

Pico projectors have not yet taken off in popularity, because they have not yet begun to be built into smartphones and cell phones. That does not matter. We have a laptop on every table, and on those laptops are browsers. To write software that can be understood by the browser, one must interact with a system of capital that is dedicated to the co-option and devaluation of the author at the privilege of the corporation (TVO interview, Sawyer). The point, then, as a practitioner of a form of creativity that has not yet been completely co-opted - for the creativity of someone solving a problem within code, a specific problem especially, is still a creative practice - is that we may resist and open a door to further resistance. 


These negotiations around copyright, around patent law and inclusion, pull back a layer on the systems which build the internet. The systems are built by large companies which buy smaller ones and make business decisions about how to include which innovations based entirely on their expense, and what business risk is revealed by including a given technology. Making use of code articulates a subordinate relationship to this machine, with a key difference.

At the moment, internet technologies are functional because they are dependent on massive code-base issues, which are negotiated by corporations in much the same way as the official French language is negotiated by the Academy Fran�aise. 


 having them be invisible is an interesting problem the same way the underlying gendered assumptions in language is an interesting problem to overcome from the point of view of the legal system. It's an interesting problem. What I mean by an interesting problem is a problem that's worth spending a long time on, teasing out all the details. It's like a math problem that's made up of words. Not a word problem, a problem with the words. So, language assumes culture, Javascript assumes callbacks. That means you write it in reverse. It's sort of like being Ginger, you dance backwards in heels the whole time.

As a coder, what you write becomes real in a tangible sense, making interesting new things is difficult, thinking about what they might be is hardest, particularly alone - so, how do you fix this? You find a gang. A girl gang! The arts have an audience, but their audience is limited by technology [\cite{LISA notes}. If you have a particularly brilliant painter or a particularly brilliant graphic designer or an awesomely amazing musician, that is very fine, however, if they cannot record their information, or transmit it to a broader audience, then they will be broke. We know this is a problem. The next problem is how can we line up the technology so that they don't go broke after they do have the audience. We're still working on that one. It's a scary one. 

Technology has a profit margin, but technology, because it builds on itself in a recursive manner, after education and privilege has given people the ability to write it, often has a very limited cultural outlook. People burn out in the tech field all the time. They burn out constantly. Google hires people for the express purpose of burning them out building machines. Because that's what we do. We write documents that are machines. They're laws. They're just laws made of a different kind of language. And they are breakable. There's an entire body of people who work breaking the laws of the language that we write into our documents. We call them hackers. They're very valuable. We need people who can break the laws to show us where they go wrong. 

Together, good art and good technology can make good experiences. What I mean by good experiences is experiences that ask questions or display well-rounded characters or use any of the many guidelines and theory that we have that drive the Humanities and the arts more broadly. If we pair people off, then we lose the part where the technologists are scared of the artists because the artists speak a mysterious language that the technologists do not understand, and the artists can learn that what looks like magic - a beautiful glass panel that just does what they say - actually takes a lot of work, it's not work that is unquantifiable either. It's built in code commits and checkins, you can see where the period goes, it isn't being a wizard. On either side. It's being a mechanic. It's assembling things well and putting them together and bolting them down and recording that so that it can be done again by another person. This is true on both sides of the equation. Feminism is equality. Art is technology. 

By driving with art, rather than technology, the experience can be newer, and less expected. This is because of the aforementioned recursiveness of technology. Tech builds on tech, almost unquestioned, but art builds on culture and culture has a way of slipping around. When you are not looking at it, it moves in the dark. You think that teddy bear is sitting on your shelf but you wake up and it's at the end of your bed and let me tell you, you will scream. Culture can be seen as nothing but questions, questions like "How might a person be?" I think the answer is better. 