% Chapter Template

\chapter{Industry Engagement and Design Research: Community Collaboration in Software Development} % Main chapter title

\label{Chapter4} % Change X to a consecutive number; for referencing this chapter elsewhere, use \ref{ChapterX}

\lhead{Chapter 4. \emph{Industry Engagement and Design Research}} % Change X to a consecutive number; this is for the header on each page - perhaps a shortened title

%----------------------------------------------------------------------------------------
%	Industry Engagement
%----------------------------------------------------------------------------------------
\section{Industry Engagement}

%-----------------------------------
%	Miso and Bento Box
%-----------------------------------
\subsection{Bento Miso and Bento Box}
In order to run a game jam, one requires space, and people interested in working on games that are in line with any themes you might want to test. In order to access that space, I worked with the Bento Miso coworking facility here in Toronto.

Miso is a not-for-profit bricks and mortar site that serves as home for both Bento Box, a local development hub, and the Dames Making Games, a feminist initiative to introduce women to digital game development processes. As a board member of DMG, I have repeatedly witnessed the limitations of extant gamemaking tools. The software has bugs, runs on only a few systems, or relies heavily on metaphors and software constructs that are understood to those who already play a range of commercial digital games, but which are not clear to those of us who are new to gamemaking practice.

As a not-for-profit, Miso also serves as the hub for a great deal of Toronto's indie - independent - game development community. They offer professional support and development advice, and I felt there was a good match between their professional skillset and my research interests. The DMG traditionally run a jam in November, and felt that screenPerfect - a new software designed to be accessible in a short time frame to people with extant skills - would be a good match for the audience associated with the organization.

Miso, and Bento Box, offered to help me with coding a more accessible front end to the screenPerfect engine in time for the jam, so that I could get feedback on the system mechanics rather than just the interface.

%-----------------------------------
%	Dames Making Games
%-----------------------------------
\subsection{Dames Making Games and Game Jams}

\begin{quote}
Dames Making Games (or DMG Toronto) is a non-profit community organization based in Toronto dedicated to supporting dames interested in making, playing, and changing games. In short, we want to build an \textbf{inclusive} and \textbf{engaged} local community of game-makers. Our community isn't women only, but it is women-driven.
\textit{from the DMG.to website, accessed \cite{dmgto} November 27 2013}
\end{quote}

The Dames Making Games are a society within Toronto that work to promote women in video games. Initially funded in part by \textbf{FiG} and part of an SSHRC grant, the DMG are now entirely self-funded. They work by using the game jam method to introduce women and allies to simple game development tools. This provides a straightforward introduction to concepts of computer logic and programming problems for some people, to video game art development for others, and video game sound production for still others. Some develop system mechanics, some design whole levels or game narratives. 

The point of the DMG is to promote access to this field to people other than the 18-to-35 year old males who form the primary demographic for the video game industry, in the hopes that a diverse population of game makers will produce a diverse population of games. 

%-----------------------------------
%	What is a Game Jam
%-----------------------------------

\subsection{Game Jams, A Design Method}

A game jam is a variant on the hackathon, which is a type of prolonged effort at taking an idea from concept to finished product in a limited period of time. They are related to design charettes or \textit{parallel prototyping} (\cite{charette}), a method whereby participants rapidly prototype a design idea over a short, intense period of time. A jam - or hackathon - gives registered participants a common area and space to set up their own supplies, and a theme. The group members come to the event with an idea and possibly some resources - video files, sound capability and so on - and use the jam time to assemble a game.

Generally, a game jam will produce a panoply of small game ideas with fleshed mechanics but simple art and sound design in order to demonstrate a possible path forward for a device or piece of software, which will then be polished at a later date, and presented to the indie community either online or at a social event. Sometimes these works will then go on to be finished commercial products, or intended for further consumption at major conferences such as Indiecade or GDC. These conferences ideally further the careers of the developers by providing access to funding bodies: publishing houses, or in Ontario, the Ontario Media Development Corporation.

Game jams can be time consuming to prepare, as they involve a great deal of communication on the part of the show-runners. In order to run a jam, one must open the application period well enough in advance to ensure a large available population of skilled users who are likely to be interested in producing content with the available tools, or interested in exploring new tools on offer. Typically, jam members have a theme suggested - "Mother May I" or "Snacktember" being a few run in 2013 by the Dames Making Games - and then participants bring their own preferred technology to knock out a fast prototype over a weekend.

\subsection{NoJam 2: Video Video}

The DMG have a great deal of experience running jams, and therefore, I partnered with DMG/Miso to get access to a group of skilled animators, filmmakers, and gamemakers. By partnering with them, I gained ready access to their community population, and they gained access to my software. One of the most common difficulties with game jams is that the short timeframe can cause a lot of frustration to new non-programmers: they spend a lot of time wrestling with tools, rather than generating the content of their games. The DMG would like to make it more straightforward for their membership to generate games and interactive narratives in a short period of time. 

No Jam is a two-week jam scheduled by the DMG in November. In order to prepare screenPerfect for the jam, I handed over the basic engine to Bento Box - the production arm of Miso - who cleaned the interface elements and released a web-based version of the software for users. This was a win for them, as they were able to refactor my local code base to take advantage of a new language they have produced, called Daimio. Daimio, being a dataflow language, is ideal for describing choice patterns as they relate to a database. ScreenPerfect is a good engine match for types of games that rely on interactive choices. 

As a pair, Dann Toliver - architect of Daimio - and I worked together to clean up the javascript elements of screenPerfect for speaking to the Daimio dataflow language. The group then released a refactored version of the code in time for No Jam, so that our participants could get a clean version of the software to work with. This was challenging for me, as it involved a great deal of trust, and moved the software away from how I had initially envisioned the UI. In particular, we needed to scrap an early idea for a branched narrative "tree" display, which was not included, although it had been planned all along.

After we received No Jam applications, we went through to choose participants who seemed interested the theme and the software restrictions, sent out acceptances, ordered food, and generally set the dates. Applicants were provided diaries to record their working process over the course of the week. The first weekend of the jam consisted of workshops from a variety of specialists to provide direction in how to think about the software and the jam process as research. My presentation is included in Appendix C, consisting of how to work with the screenPerfect software, how to think about multi-screen video, and how to think about technology as a form of creative practice which is both limiting and freeing. 

The applicants were then sent home for a week to work on their video projects, and asked to document their ongoing process with one another on a private Google Group. Most participants ignored this request, which left us with relatively little promotional material.

On the actual weekend, we asked that participants arrive with the majority of their video content and design prepared. There were vastly uneven responses to this request, which strongly affected the ability of participants to produce a finished game by the end of the weekend. I interviewed each group early in the process, and then later polled them with informal questions regarding their experience with the software.

The group experience with the software proved interesting. Accomplished filmmakers had a better time with it, but the most suprising response was from young, self-identified gamemakers, who rather than exploring what was possible within the context of the software tools, decided instead to try to use them to reproduce existing game types, many of which were totally incompatible with the software's design. Of particular interest was the group who tried to reproduce a classic Japenese roleplaying game within the context of video: this did not work so well, and they continued to work at it even after it became apparent it was unlikely to go well. The game itself remains unfinished, but deserves mention as the most unique and possibly stubborn effort. Used to working with uncooperative tools, the participants seemed unsure how to cooperate with a tool clearly designed to a single end.

Despite this surprising result, No Jam was a success, with nine groups producing diverse works on ideas such as how to express a practice of mindfulness, how to work with pornography in a way that forces the viewer to interact with what's happening on screen, exploring systematic violence against women, exploring narratives of imprisonment, magic, and in one unique case, permitting a puppet to escape a toy box. 

In setting up No Jam, we did present at least one workshop on the importance of the personal narrative in producing creative work, which may have influenced the results. Game jammers mostly described their interest in producing work that was finished, and one jammer explicitly stated that she was pleased to have had a finished work at the end of the jam, this being an uncommon result for her when she had to learn the usual round of new software each time.

No Jam resulted in at least five "finished" works, which have since been included in several exhibitions around the city, including the December and January Toronto Long Winter series.


%-----------------------------------
%	Software Design for Game Jam
%-----------------------------------

\section{Software Design}
\subsection{Interfaces, engines, and interactions}

A software interface is the part of the software that a person interacts with directly \cite{interactiondestext}, where a software engine is the part of the code that detects and defines what a computer can do with that interaction. The engine that I coded responds to interactions sourced from users. Interface interaction is what appears to to define the majority of user experience, but the response of the engine underlying that interface is just as important. A key part of the design of screenPerfect is that it was laid out to handle things like auto-saving invisibly, so that a user's work would not be lost.

The interface of the software is just as important as the engine, however, because a poorly designed interface will confuse a user, thereby rendering the experience of using the engine figuratively opaque. screenPerfect's roots are as a software engine, which takes user interaction and then does things with it. The user interacts with the interface, which speaks to the engine, which then returns values to whichever interface the user has selected.

\begin{figure}[h]
  \caption{screenPerfect software communication model}
  \centering
    \includegraphics[width=0.5\textwidth]{SPmodel}
\end{figure}


\subsection{Initial screenPerfect Engine|Interface Layout}
In the case of screenPerfect, the interface is laid out in three parts. The first part is the setup screen, which is where game designers load their media (both videos and static files) and lay out the links between those files. This is the essence of a game made in screenPerfect: which choice will a player make to navigate the system as designed by the artist?

The further screens are the client and control screens. screenPerfect supports up to ten client screens and ten control screens, although the interface only exposes a polyphony of client windows, while restricting artists to a single control set for simplicity's sake.

\begin{figure}[h]
  \caption{screenPerfect initial screen layout. Client, Control, Setup.}
  \centering
    \includegraphics[width=0.5\textwidth]{threeclient}
\end{figure}

\subsection{screenPerfect layout: NoJam}

As the chief author of the screenPerfect software, it is very easy for me to understand where video files should go, and I come to understand how the software works implicitly. This is a common problem in software authorship, much as text requires editing and paintings require critique. I brought the software to Bento Box for refinement for reasons already stated: they needed a tool to express use cases for Daimio, and I needed help designing a UX that was actually useful to users.

The final layout of screenPerfect is much cleaner than the one with which I had been working. Rather than hidden tabs, everything is laid out clearly, and allows users to see where their files are going. It is open-sourced, and available on GitHub for evolution by advanced users - the DMG has already forked a version, iV (\cite{iv}), for further development.

What follows are screencaptures of the pre-fork game jam variant of screenPerfect.

\begin{figure}[h]
  \caption{screenPerfect NoJam screen layout. Client, Control, Setup.}
  \centering
    \includegraphics[width=0.5\textwidth]{threeclientNoJam}
\end{figure}

%----------------------------------------------------------------------------------------
%	Design Research
%----------------------------------------------------------------------------------------
\section{Design Research: What Goes Into Starting A Software Project}

\subsection{Artist Collaboration}

Beginning development from a first position within the arts is unusual, even for an Agile workflow, but in the case of this software, it has been wholly driven by artistic collaboration. I firmly believe that simple tools to relieve the friction points of the artistic process will lead to better art which is more widely available. This is to say, although the developer is themselves a creative who will decide \textit{how} to solve problems, the problems to solve may be better handled by an outside party. This is down to an issue of demand. 

In order for software to exist, and to be seen to exist, it requires an interaction. Unlike a hammer, which takes up space on a shelf, software is essentially a text document until it is used. As \citeauthor{galloway}says, a game is defined by action \citeyear{galloway}. \cite{galloway}

%----------------------------------------------------------------------------------------
%	Agile Methodology
%----------------------------------------------------------------------------------------

\section{Development Methods}
\subsection{Agile}
The Agile methodology is based on a manifesto, as is so much else of this work. Agile is a response to previous software design practices, called "Waterfall," where software frameworks are laid out and heavily documented in advance of production. Waterfall methods are popular in major software companies, which rely on extensive documentation to communicate between business units. They emphasize planning over software production or delivery deadlines. 

Agile, described in 2001 by a group of software developers, reflects a less top-down approach to the software development practice. Rather than pre-planning every element of software, screenPerfect was designed by discussion with key stakeholders as to how it should come together, and what the final product of the development process should be. This is a hacker-oriented means of development, reflecting Plant's statement that reverse engineering - "starting at the end, and then engaging in a process that simultaneously dismantles the route back to the start" is how hackers work \cite{plant}. Agile, released four years after Plant's assertion, engages the process of iterative development via reverse engineering in a more formal sense. Agile specifically emphasizes \textbf{individuals and interactions} over processes and tools, \textbf{working software} over comprehensive documentation, \textbf{customer collaboration} over contract negotiation, and \textbf{responding to change} over following a plan. 

In the case of my development process, this worked as follows: The initial project was laid out by Hannah Epstein, who described how the game processes for psXXYborg should work using a series of YouTube videos linked through their annotation technique. In development conversations, it became clear that Youtube, in addition to having many distracting advertisements, also did not have a great way to make annotation invisible, and was very slow to load. This is a problem with reliance on external networks: they cannot be as fast as locally served files. Hannah specifically emphasized speed, smooth loading, and using video based in static rather than streaming or live files. These needed to be served within a closed environment to an attentive audience.

From that point, I began to research languages that emphasized speed over structure. Scripting languages, without strong classes or inheritance, are especially good at this type of development work. I reached out to other developers and asked how they would solve this problem, and they came back to me with a variety of answers - some used PHP, some used Python, all of them relied on JS for their front end. In researching different ways to solve the basic problem - passing a variable back and forth through wireless technology to select two on-screen videos at almost the same time - I discovered the Node framework. I explained how this would work - minimum installation time and expense, reliance on a straightforward machine installation on-site - and got approval from my partners.

At that point, I went home and wrote a server using Express.JS, socket.io, and node to write an application intended for the Safari browser, in line with my partner's habit of exporting video in the restricted H.264 format. Due to conflicts relating to codec patenting, one of the many secret things that underly the "free" internet, Safari supports H.264 where Chrome does not. Chrome supports Webm via the V8 engine, the same engine that supports the Node framework. Webm is also a more compact video format, which results in smaller file sizes and lower bandwidth costs, which eventually affects both load time and playback lag on client machines. 

Having worked extensively to serve H.264 video and discovered that loading many videos in H.264 will quickly overload the Node-native server, I reencoded the video works into the Webm format and converted my development process to pursue the Chrome browser. This also made psXXYborg available on mobile browsers on the Android platform, which is advantageous, as Android is much more readily available to low or no-budget projects than Apple devices, even old ones.

Throughout this process, I would meet with Hannah, who would provide me with updated feature requests. This is normal for the agile process, which places an emphasis on completing software delivery by deadlines, rather than on documentation and a firm attachment to original plans. 

In order to keep track of all of the changes to the software, I used the GitHub source control system. GitHub allows users to store and update their code base while keeping track of changes in what are called "commits." Other GitHub users can then "fork" or copy a specific version of the existing codebase, and from there make their own changes. A "fork" is considered a new project. A "branch" is a "fork" within a project that contains changes not yet passed to the "master" branch. My GitHub commit log for the screenPerfect fork of psXXYborg can be seen in Appendix C, which documents the changes made over months - bug fixes, design shifts, changes in the layout of the program, and new ideas for the setup and control files, including the late inclusion of how branching narrative would actually work without a database.

Overall, Agile worked for this process by allowing me to respond to user requests for code changes and information rather than forcing me to work to a standard pre-set from above. A Waterfall process would have discouraged me from even attempting to take on the work. By working in small steps back from a pre-set destination with total freedom as to how the code actually came together, Agile allowed me to demonstrate different working parts of the software as they came together. The documentation for the project is tied into the code commits, and inseparable from the actual written code within its archive.

\subsection{GitHub and Open Source Software}
The development of screenPerfect is dependent on a variety of external technologies. Although relatives and derivatives of Google's V8 system are foremost among these, there is also a dependence on the licencing and mindset of the open-source movement, and the GitHub software repository system.

Open source software is not the same as free software, as provided by structures such as the GNU General Public Licence . Open source is the peer reviewing system of software. What open source means is that even if a given piece of software compiles to a single program which can then be distributed for use on the desktop - as screenPerfect does not - the code that goes into the executable file is freely available on the internet, to be changed, supported, and developed by the population of software workers who exist in the broader world. These developers may work on closed or open source projects in their usual working time. They may be very skilled or quite new to development work. What matters is that the software's code is then shared publicly, where it can be reviewed and compiled and extended and changed by anyone at all.

The intent of open source is that anyone may learn from such freely-shared information, and anyone may contribute to the collective knowledge base. In this, GitHub is not unlike JSTOR. The stark difference is that GitHub is costly to the user who is sharing information, rather than to the user who is seeking information. A public GitHub account costs nothing, a private one with a limit on projects is less than $10CAD per month, and any public projects can be found online by anyone. This is useful for reference, as one can quickly find the solution to many coding problems for the cost of looking them up, and an internet connection on which to do so.

There are some obvious problems with open source. One of the clearest is that with all that intellectual property out there for free, it is a challenge to make any money on an open project. The other is that there is no way to guarantee quality: one takes what one can get, although it is assumed that contributions to projects are made in good faith, and major project contributions are checked by trusted individuals before they are published. For example, the Mozilla project relies on contributors whose code is applied to the codebase after approval by certified reviewers \cite[URL]{mozillacontribute}. 

\subsection{Licencing}
One of the ways these problems are dealt with is through licencing. The Creative Commons at creativecommons.org expresses their mission as follows: "Creative Commons develops, supports, and stewards legal and technical infrastructure that maximizes digital creativity, sharing, and innovation." It is therefore an appropriate open standard licence for \textit{creative practice}. A preferred licence for software development is the MIT Licence, which is closer to the Gnu Public Licence, but does not preclude making money from one's open source work.

\subsection{Science Fiction Inputs}
My own idea for how this project would work is taken from Cory Doctorow's \textit{Pirate Cinema}, which features a scene wherein characters climb trees, and using pico projectors already built into their phones, assemble a movie theatre from nothing more than sheets and ropes in the trees. I felt this sort of mesh-networked sharing is much more likely than a continued reliance on the surveilled internet for sharing copyrighted and copyrighted-material derived works. Since I could not find a system that would permit this type of sharing on the internet, I felt that this project would provide a good chance to build one.