% Appendix C

\chapter{Agile Manifesto} % Main appendix title

\label{AppendixC} % For referencing this appendix elsewhere, use \ref{AppendixA}

\lhead{Appendix C. \emph{Agile Manifesto}} % This is for the header on each page - perhaps a shortened title

\section{Agile Manifesto}

Manifesto for Agile Software Development

We are uncovering better ways of developing software by doing it and helping others do it.
Through this work we have come to value:
\begin{itemize}
\item[\tiny{$\blacksquare$}]  Individuals and interactions over processes and tools
\item[\tiny{$\blacksquare$}]  Working software over comprehensive documentation
\item[\tiny{$\blacksquare$}]  Customer collaboration over contract negotiation
\item[\tiny{$\blacksquare$}]  Responding to change over following a plan
\end{itemize}

That is, while there is value in the items on the right, we value the items on the left more.
© 2001, agilemanifesto.com

\subsection{What Is Agile?}
Agile is an ideal based on a manifesto released at the turn of the century. It is not specific enough to serve as a development framework, but instead serves to structure a way of thinking about software development in collaboration between developers and their user population. The original Agile Manifesto was released in 2001, and has a vast number of signatories to date. 

Agile is intended to address the gap between planned software, which is typically fixed at release, and reality, which is that software breaks routinely and needs regular maintenance and iterative upgrades. The manifesto suggests that developers should focus on results over process, software that works over software that's well-documented, collaboration over contract specifics, and responsiveness to change over following a specific plan.

Agile has proven useful for this project because its emphasis on responsiveness and deliverables has ensured that the working software of screenPerfect can move forward to permit more games to be made, rather than locking the software to a concern for itself as an object. This is a valuable way to move through a development process which spares us from needing to commit to a given method, and instead emphasizes results from practice.

This practice must be documented, which is where the code-commit process appears. Agile deprivileges documentation for itself, preferring to include the necessary as the code itself is logged. Therefore, I have chosen to use the Git toolset to retain my version control on both this document itself and the code developed to run the main game engine.