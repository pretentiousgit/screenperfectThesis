% Appendix F - Lisa Notes

\chapter{Notes from Leaders in Software and Art 2013} % Main chapter title

\label{LISAnotes} % For referencing the chapter elsewhere, use \ref{LISAnotes} 

\lhead{Appendix F. \emph{LISA 2013 Notes}} % This is for the header on each page - perhaps a shortened title

%----------------------------------------------------------------------------------------

\section{What is LISA}
LISA is a two-year-old conference hosted in rotating locations, designed to bring together software art practitioners from around the world. In practice, it is focused on New York City, although it does showcase a broadly international range of work. In 2013, LISA took place at Parsons: The New School on Friday, November 1. The showcased works were interesting - a lot of work done by artists interested in OpenFrameworks and OpenExhibit - but more interesting was the opening panel, which focused on storing and collecting new media works. There was a small amount of effort made to present 3D printing as an exciting new endeavor, but this mainly showcased new manufacturing techniques and MakerBot, an already successful New York startup lead by Bre Pettis.

\section{From Virtual to Real – Successfully Translating Digital Work to a Collection Context}
Moderator: Christiane Paul, Adjunct Curator of New Media Arts at the Whitney Museum and Associate Professor, School of Media Studies at The New School.
Panel: Lev Manovich, Lynn Hershman Leeson, Magda Sawon, John F. Simon.

Notes on panel:
-- Labelling and presentation of new media works is difficult in a gallery context.
-- Complaints about TMS' labelling system and storage/recall of works
--- TMS is a custom curatorial inventory database system widely used in museums around the world. It is private and does not readily link to the internet except via the eMuseum interface.

-- Selling an art website may include negotiations for things like the hosting, the domain name, and maintenance fees for these things outside of the work proper.

\textbf{John F Simon}
-- Twenty years ago [we] wanted an app store, because it [is] so difficult to get money back out of art software.
-- The concept of the art may be great, but there is a difference between concept and physicality of piece.
-- People like having a \textit{thing}. The thingness is important.
-- Tahe advantage of the new as it's new, when you first see it. 
-- Simon has moved to CNC from software because of the difficulty of making software into a thing.
-- It has become rarer to have a CNC tool than a screen. Screens are no longer special, but a custom wall is.

\textbf{Lynn Hershman-Leeson}
Hershman-Leeson is an artist who has been working in New Media since the mid-1960s. She has made many projects using branched video. 
-- Hair, machines from 1963 
-- Sound and breathing was shut down because "not art."
-- Loma 1982 - made choices about the protagonist's life direction. 
- Made surveillance systems, behavioural models based on the stock market in 2000-2002, made Agent Ruby in 2002.
- DiNA AI in 2004 - a biological art piece, like with Incubator and Jen Willets.
- Present Tense is a genetics lab about DNA being an archive.
- She didn't have a lot to say other than that her art is deeply unpopular for collections because it's too difficult to maintain or display.

\textbf{Magda Sawon}
Runs galley in TriBeCa to sell new media art. "Media Neutral" - does software and oil paintings.
-- Flatly states that screen-based media does not sell. At all. She figures that 3,000 years of making objects works against the ability to think of a non-physical thing as having a direct value.
-- Therefore, you MUST make a box. It MUST be commodifiable, because otherwise it simply doesn't sell.
-- Shows Wolfgang Staehle's 2001, which was a live transmission of the WTC from 6/9/01 to 10/10/01. 
-- This was a really BIG documentation
-- Intended as a landscape in line with Staehle's other works, it became a painting of history.
-- Photography departments are more open to this kind of acquisition, as they have more experience with them.
-- Lenticular screens as paintings - a way to commodify the GIF. http://gifpop.io/

\textbf{Lev Manovich}
-- New book, Software Takes Command
-- Cover is Kingdom Hearts being played over 62 hours
-- What is a video game? Take this recording, but the CONTENT of the game is actually infinite, as it changes slightly with every play through - no-one does it perfectly similarly each time.
-- Big Data can therefore be a single game played differently.
-- Explore the possibilities and affordances of the media.
-- Impossible to say how many media we have, as the properties of a piece are no longer contained within an object.
-- Not meaningful to distinguish properties of different media if it is all produced by software.
-- Waiting for tech to catch up.

\textbf{Panel as a whole}
Agree that technology art takes a long time to be acknowledged, which means that we just have to do the work and wait 40 years. No big!
Note: This was a pretty cynical panel, and what I took away from it was that ScreenPerfect needs to be a cartridge, not just a program.

-- Work does break and it needs maintenance.
-- Include portfolio for making, include a 30-year guarantee on the work.
-- Museums have trouble storing the manuals.
-- Museums are inflexible.
-- Majority of software works are hard to keep.
-- A fix could be for museums to hire younger people who aren't uncomfortable with technology.
-- Pretty much all of the works shared in the panel are already broken.
-- How do you fix broken works?
--- buy backup laptops and computers for parts.
--- John F. Simon used Apples because they were so prevalent he figured there would be a parts supply for decades, which is almost as good as custom ROM.
--- There is a market in maintenance and repair, conservation.
--- Software is not perpetual.

\subsection{Do you sell 3 or 3,000}
-- Artificial scarcity permits an increase in price.
-- Cost of cultural history is to be considered.
-- Few museums will pay for things that are not film or video based.
-- Time-based work is not available to audiences that will not take the time to consume it.
-- Lev Manovich says either we're all dead or museums catch up. Photography took 140 years to be art.
--- We don't make things fixed, but variable.
-- Institutions must renew.
--- Silion Valley is the obvious choice to support this work, but they won't do it.

\textbf{Q&A}
Archivist asks how artists preseve their works when museums won't.
-- Keep your records well.
-- Ask the Tate or SF-MoMA.
