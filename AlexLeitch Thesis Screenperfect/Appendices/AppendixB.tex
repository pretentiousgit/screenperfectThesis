% Chapter 2

\chapter{Annotated Literature Review} % Main chapter title

\label{AppendixB} % For referencing the chapter elsewhere, use \ref{Chapter1} 

\lhead{Appendix B. \emph{Annotated Literature Review}} % This is for the header on each page - perhaps a shortened title

%----------------------------------------------------------------------------------------

\section{Literature Review}

http://kotaku.com/the-weird-escapism-of-life-sims-730629952 Leigh Alexander on aspiring to pay a mortgage in real life, which is important because people won't be able to do that, a lot, in North America. 

http://agilemanifesto.org/principles.html The Agile Manifesto, which backbones my actual software development style: working software is what counts. Important, as Agile is in direct defiance of corporate control structures.

Maly, T. (2013). We Have Always Coded. Medium.com. https://medium.com/weird-future/2acc5ba75929
	This article investigates gender essentialism from the perspective of biological essentialist arguments frequently used to say women can't or shouldn't code because of various, entirely specious, evolutionary problems. This is an argument that happens on top of a perceived resource scarcity; the scarcity is in this case employment of women in technology, so opportunity.

Fashion article about hegemony of fashion writing and the death of exclusivity, with notes on shows being locked down to resist the blog invasion. http://thenewinquiry.com/essays/cool-fronthot-mess/[I need some work on how basic economics works with supply and demand in the absence of a good money supply.

bell hooks. (1992). Is Paris Burning?. Black looks: race and representation (pp. 145-156). Boston, MA: South End Press.
	bell hooks is always useful in concert with Derrida to explain that you can't actually know what anyone else is actually thinking; having sympathy for other people is not the same as understanding their direct experience. Useful because it underlies a lot of feminist practice. This is an article about exclusivity, which is an issue of privilege, which is an issue of perceived resource scarcity, in this case access.


Bizzocchi, J. \& Tanenbaum, J. (2011) Well Read: Applying Close Reading Techniques to Gameplay 
	Experiences. In Well-Played 3.0, Drew Davidson Eds., Etc press
	www.etc.cmu.edu—well-read-jim-bizzocchi-joshua-tanenbaum
	Something to explain game design processes in the technical section of the document.

Buxton, W. (2007) Sketching User Experiences: Getting the Design Right and the Right Design. Morgan Kaufmann Publishers. 
	Something to explain user interface design practices in the technical section.

Chun, W. H. (2011). Invisibly Visible; Visibly Invisible and On Sourcery and Source Code. Programmed visions software and memory (pp. 1-54). Cambridge, Mass.: MIT Press.
	Still need to read this, but the title is pretty relevant to the central research question of my thesis, which is on the value of code and invisibility as a permission.

Cixous, H., Cohen, K., \& Cohen, P. (1976). The Laugh Of The Medusa. Signs: Journal of Women in Culture and Society, 1(4), 875.
	Woman must write her own desires into being in order to be seen. This is a paper that reinforces arguments about scarcity of perception, and issues of control, specifically of women and women's opinions.

Deleuze, G. (1992). Postscript on the Societies of Control. October, Winter(59), 3-7.
	Surveillance culture is bad for people, yet inevitable as it becomes automated. This is an issue of control, which is subverted by taking possession of even a single means of production; to refuse to code is to be illiterate of the systems by which production is governed. To refuse to acknowledge the implicit control of a system is to lie to oneself; if other people have designed the system, the system operates[Alex Leitch, 2013-10-01 2:18 PM
There are no complete systems, particularly in computers, because Godel's incompleteness theorem says so. This is a joke that is also true and I am not sure how to cite it, but it holds for most systems of even informal logic, which computers are composed of. This is a purely theoretical way to look at a computer system that is both true and not true; the incompleteness theorem concerns math systems, not people, specifically. You can't test people for their incompleteness. But people are still incomplete. If they weren't incomplete, they wouldn't have religion.] as it is intended. The way out of this is to read the system, then find the gap within it. 

Dell, K. (1998). Contract with the skin: masochism, performance art, and the 1970's. Minneapolis:University of Minnesota Press.
	Pain or revulsion is another way to escape a system, which is to make it so dear - dear as in price - that it is difficult to reproduce the work because it costs too much. Abjection, or the ability to pay for something with revulsion, is one of the less efficient but more effective systems of resistance. The liminal space represented by a willingness to publicly maim oneself is reserved for those who do not fit well within a system that relies on completion: broken skin stands in for a refusal to submit to hierarchy. 
This collapses in various ways over time - it's unlikely that even facial tattoos will keep people out of the workplace for long - but still represents a real way that people refuse to be included in a more perfect/uniform work. This is an article on how masochism, the revealing of the inside, has a recent history in art and what role that sort of display performance contributes to feminism.
	

Doctorow, C. (2008). Little Brother. New York: Tom Doherty Associates.
	A detailed look at how surveillance culture breaks down social contracts, and a how-to guide on resistance via action disguised as a novel. Also has a variety of excellent, accurate examples of ways to disguise data so that it cannot be confirmed by a rogue authority. Connected to Deleuze on Societies of Control, and specifically addresses homeland security spy tactics.

Doctorow, C. (2012). Pirate Cinema. New York: Tom Doherty Associates, LLC.
	Contains a scene with multiple tiny projectors used to set up a cinema in a park from pockets, which is more or less what the software itself is supposed to do when it runs. The entire book is about resistance to copyright authority. Contains a quite didactic passage on how even hardware control chips do not actually control or prevent smart-enough people from using controlled software, and in doing so, presents a vision of the world where the most privileged are no longer privileged with money alone, but also with knowledge or access, which is a type of prestige - which is a type of magic trick. Concerningly libertarian.

Gleick, J. (2011). The information: a history, a theory, a flood. New York: Pantheon Books.
	A survey text of the history and development of data-centric information technology. Explains a little of the context for how tools like screenperfect can be expected, themselves, to proliferate to the point of uselessness. This is useful because it prevents needing to look up each paper about completeness theories and map-rename signal-to-noise mathematics independently. Signal-to-noise mathematics are important because they provide a way to think about how to privilege information in the learning process, or the internet search process; people passively look up how to find what they need. Downside: The noise often contains trace characters that allow further exploration. Upside: there really isn't that much signal out there, no matter how much noise is happening.

Gram, S. (2013, March 1). Textual Relations: The Young-Girl and the Selfie. Textual Relations. Retrieved 
	April 12, 2013, from http://text-relations.blogspot.ca/2013/03/the-young-girl-and-selfie.html
	Young women's bodies are not only super-powerful, but can be the stereotype vehicles for all of consumer culture. This is important because young women are disproportionately discouraged from tech culture, even as they control the scarcity that is approved sexual relations within North America, a scarcity that cannot be overcome with money alone - you can't buy love, they say. So this is valuable because it is an excellent analysis of how stereotypically correct bodies benefit from fitting into a system of action, which is related to code practice because only stereotypically correct bodies are encouraged to participate. Per masochism, however, there are no correct bodies, only correct images of bodies. Resist the system in a predictable direction, and you become a new format of marketed body, with a new predictability. This predictability can be coded, but the closer one gets to perfect, the harder it becomes to occupy the role while remaining human. This is a deeply misogynist text, but is a perfect text for examining the role of image on the internet, and things which are seen but hard to describe, as code is.
	

Grosz, E. A. (2008). Chaos, Cosmos, Territory, Architecture. Chaos, territory, art: Deleuze and the framing 
	of the earth (pp. 15-28). New York: Columbia University Press.
	Technology as sexual performance and definition of space. Not terribly well-realized but more academic than other sources on the same subject. Useful because code is a creative process, and creative processes - per Wilde - are useless … like peacock feathers, or any other sort of look-at-me performance. Even things which do things are useless.

Haraway, D. (1987). A Manifesto For Cyborgs: Science, Technology, And Socialist Feminism In The 1980s. 		Australian Feminist Studies, 2(4), 1-42.
	Primary text on women restructuring their bodies, invisibly, to take over the world. See also Quinn Norton on IUDs (http://www.quinnnorton.com/said/?p=404) - this is useful because women can resist commodification, such as that described by TIQQUN, invisibly. Code is, in Agile practice, shifting from architecture to a sort of cooking; this library and that, all put together in a frame to pursue an idea, rather than to do a specific thing from the outset. This is a text about resisting control systems by allowing oneself to cooperate until there is a space to break free. Frequently, people don't even notice you have.

Haraway, D. (2009). The companion species manifesto: dogs, people and significant otherness. Chicago, 		Ill.: Prickly Paradigm Press.
	More Haraway. Now on cancer, not sex. I need to read this but I don't think it will be too useful, except that it articulates that humans, with their tool-use, are not actually special; we are part of a system of mammals. This may be useful elsewhere.

Hunicke, R., LeBlanc, M., Zubek, R. (2004) MDA: A Formal Approach to Game Design and Game Research. sakai.rutgers.edu—hunicke_2004.pdf
	More on game design techniques for the technical portions of the paper. Describes methods which will need to be addressed as part of the Agile/How Did I Make This Game methodology.

Kristeva, J. (1982). Powers of horror: an essay on abjection. New York: Columbia University Press. (http://www.csus.edu/indiv/o/obriene/art206/readings/kristeva%20-%20powers%20of%20horror[1].pdf)
	Horrifying things have a power that is more potent than any non-horrifying things could hope to possess. This paper details why that is. It goes very nicely with Cixous and discussions of the IUD, because it is about what happens when barriers truly break down. I think Kristeva's horrors are basically the key to the entire news cycle and Grand Theft Auto to boot. This paper is the original on how revolting things are fascinating but resist being part of a system, unless they're cleaned away and perfect. See above comments on masochism paper; the awesome attraction of the awful.


Krug, Steve (2000) Don't Make Me Think: A Common Sense Approach to Web Usability.
	Riders Publishers.
	This is another technical paper for arguing that software design should be totally invisible. Useful because it ties together the systems of control argument - control is implicit, presented as undefeatable, a smooth surface - with the idea that things should be useable, so that people can find their own uses for the tool beyond what is initially intended by the author.

Schafer, T. (1998). Grim Fandango (1.0) [Video Game]. USA:LucasArts.
	Classic adventure game with minimal interface and a fixed runthrough. One of the last great adventure games. An excellent exercise in game design where the game itself is pre-set, but the ideas the game displays, including an interest in a subculture that is not much popularly examined (Mexico), and a good narrative. Evidence that narrative is important in gameplay, which is key to the development of screenperfect as a narrative branching tool. Also remarkably and incredibly broken on contemporary systems, as the initial code was rendered directly by processor speed, rather than at a stage or two removed; the game was broken by Moore's Law, which is good evidence for why tools need to be considered unto themselves. The new narratives are temporary. This is a narrative about the temporary; death, and the waiting period before leaving - while being wound up in a longstanding celebration.

Luvaas, B. (2006). Re-producing pop: The aesthetics of ambivalence in a contemporary dance music.International Journal of Cultural Studies, 9(6), 167-187. Retrieved April 10, 2013, from the 
	Scholar's Portal database.
	An interesting look at what ethnographic research can be, and the speed of cultural shift and recycle since the rise of the internet. To be read in concert with various VICE mag articles about cocaine, new york. Used originally in article about Seapunk movement.

Moggridge, Bill (2006). Designing Interactions. MIT Press, Cambridge MA.
	More technical reading about how people interact with software, about how people can control interactions.

Moyer, J. (2012, September 14). Our Band Could Be Your Band: How the Brooklynization of culture killed 
	regional music scenes - Washington City Paper. Washington City Paper - D.C. Arts, News, Food and 	
	Living. Retrieved April 22, 2013, from 
	http://www.washingtoncitypaper.com/articles/43235/our-band-could-be-your-band-how-the-brooklynization-of/

	Cultural uniformity because the internet makes things from different places seem the same, even though they're really not the same. Relates to the Young-Girl article about how if you are one perfect shape, that perfect shape will always sell at least a little, which obfuscates the truly beautiful and interestingly specific evolutions with things which have been data-optimized to be more popular. Popular isn't better, and neither is monoculture, but also no good is the sort of individuality that is itself a sort of monoculture.

Mulvey, L. (1975). Visual Pleasure and Narrative Cinema. Screen, 16(3), 6-18.
	On the male gaze, which is the central gaze in most videogames, particularly first-person shooters.	This is important because the male gaze sets how most blockbuster video games are allowed to be perceived. Important because video games, like most software, are mainly compared to cinema, even though they have very little in common with cinema for elements beyond the technical. Core to arguments about how women are seen, which is essential to understand the TIQQUN readings in their slightly tongue-in-cheek misogyny.

One Laptop per Child. (n.d.). One Laptop per Child. Retrieved July 3, 2013, from http://one.laptop.org/
	I was thinking about discussing how the OLPC project led to various other tech advances, including the rasPI - it made netbooks happen, then tablets happened. The OLPC was the project that said "wait, things don't need to be faster, they need to be better." Absolute disaster; in the countries it was intended for, it was already superceded by mobile phones. Classic example of condescending outsiders trying to Make A Difference rather than examining difference. Probably too broad a scope for this project.

Orlan: a hybrid body of artworks. (2010). London [u.a.: Routledge.
	Orlan led to Lady Gaga so directly that she has since sued her. Discussing the liminality and limits of flesh without Orlan's surgeries is a challenge; almost no other artist (burden? Shoot) has gone so far, but this distance is collapsed in film like Nip|tuck and the normalization of Hollywood surgery. Related to Kristeva and articles about the mortification of the flesh for the sake of appearances, which is what I am interested in with the arcade box. Although I want that to be subtly upsetting, not overtly upsetting.

Reines, A., \& TIQQUN. (2012). Preliminary materials for a theory of the young-girl. Los Angeles, CA: Semiotext(e)
	The new translation, which includes a feminist preface by Reines about the body of young women and how she almost was sick over the assertions of TIQQUN, which happened about the same time as everyone else was going bananas for Second Life, a game where you make an entirely new body that has since been abandoned by all but the most escapist. TIQQUN accurately observe that people are escaping into their own bodies, not those of the computer screen; the new presentation is that the brain and image on the internet reinforce the physical appearance through the phone, a piece of technology governed by the male gaze.


Stephenson, N. (1995). The diamond age, or, Young lady's illustrated primer. New York: Bantam Books.
	This is pretty well a perfect piece of fiction about cyborgs and universal education and China as an Oriental-escape paradise. Fun look at a post-scarcity economy that has simultaneously happened and can't happen. This is a book about an alternative resistance to the always-on personal presentation future, where books reflect their users. This is a fantasia, but an appealing one, with a lot to say about the subject of veterans, what abuse looks like from a perspective other than the dominant, and what recovery might look like. The main characters are all female, and all develop in different directions, including one who escapes by using the book to hack out a new life under direct supervision. Contains an unpleasant thesis about the value of personal matriarchal influence on future leadership.

Sternberg, M. (2012). They Bleed Pixels (1.0) [Video Game]. Toronto:SpookySquid Games.
	Excellent representation of a female lead game character in a genuinely challenging platformer. Useful because it exposes the programmer's preference for difficult-but-rewarding game mechanic loops, along with a conscious choice to show a young woman who has strong personal agency as a hero. A manifesto for better, simpler video games.
	
Swartz, A. (2013). Aaron Swartz's A programmable Web an unfinished work. San Rafael, Calif.: Morgan \& Claypool Publishers.
	The internet doesn't belong to us, but it could, and here are some technical guidelines to pursuing that as a worthy goal. This is the other way to approach technical development; something that should be extended rather than presented as complete in and of itself. Swartz rebels against societies of control by describing systems to expose information at a basic level rather than obfuscate them. This eventually led to his death.

Team Little Angels (2009). Bayonetta (1.0) [Video Game]. Japan:Sega.
	What a hilariously sexist but also perfect meta-narrative of female power while subject to the male gaze. The rudest, most violent fun game released to ever feature a lady protected, literally, by her hair. A game with a strong female lead in the hilariously Kate Beaton "strong female characters" mold, which is problematic in its presentation even as it is simultaneously winking. Has unfortunately fixed gameplay goals, but allows players the reward of working through the game on a basic mechanic of style rather than skill alone. Fun!

Toom, A. (2012). Considering the Artistry and Epistemology of Tacit Knowledge and Knowing. Educational 
	Theory, 62, 621-640. Retrieved April 12, 2013, from the 	Scholar's Portal database.
	More technical information on how to design interfaces so that people understand, passively, what they're supposed to do with it. This is about passive learning, which is how most people learn software: through exposure and experience with previous systems, we understand the language that the developers no longer expose even though help systems. The way of using the software becomes implicit.

Volition Inc. (2011). Saint's Row the Third (1.0) [Video Game]. USA:THQ.
	This and the followup, Saint's Row 4. Games that took the GTA pattern and subverted it to make a game that is cleverly and strongly and messily about playing video games and the fantasies of those games. Saint's Row is a sandbox video game about playing videogames and what a videogame means at its base. It allows people to play as whatever type of character they like, which exposes the fallacy that videogames are solidly about anything but mechanics; the art and design on top expose the code in their very mutability, but there are no ways to solve the game puzzles except violence. This is entertaining, because rather than being a game about traffic patterns and random mayhem specifically (GTA-V), it is a game about playing games, about false achievements and the ability to do anything at all as long as it's violent. Also notable because first lead female character is a hacker from the FBI. This is an important plot point. You can also play gay or with the robot AI you rescue … but not the vice president, who's a dude. Central to my argument that software development is a second-stage creative practice because with no fixed skins, the game itself is much more exposed. 