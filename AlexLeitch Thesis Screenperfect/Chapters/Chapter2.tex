% Chapter 2
\setstretch{1}
\chapter{Background, Theory, and the State of the Art}\thispagestyle{empty} % Main chapter title

\label{Chapter2} % For referencing the chapter elsewhere, use \ref{Chapter1} 

\lhead{Chapter 2. \emph{Background, Theroy, and the State of the Art}} % This is for the header on each page - perhaps a shortened title
\setstretch{2}
%----------------------------------------------------------------------------------------
% A brief section giving background information may be necessary, especially if your work spans two or more 
% traditional fields. That means that your readers may not have any experience with some of the material 
% needed to follow your thesis, so you need to give it to them.
%----------------------------------------------------------------------------------------

%----------------------------------------------------------------------------------------
%	Software
%----------------------------------------------------------------------------------------
\section{Game Engines}
A game engine is a collection of software designed to make it possible for a team of artists, developers, musicians, and producers to work together to produce a complete digital experience.. Traditionally, game engines are used to produce 2D or 3D experiences using assets such as 2D sprites or 3D player character/interaction models, backgrounds, interaction assets - crates, for example - music, and scripts in a programming language to tie all of these together into gameplay. 

Some popular professional engines at the time of writing are Unity3D, which features native mobile integration and ease of scripting in both Javascript and C\#; Crytek, which comes with many high-end 3D resources preloaded for high definition graphic support; the Unreal Engine, which is quite stable and useful to experienced teams that prefer more control over their work.

There are popular hobby engines that de-emphasize programming as well, such as GameMaker, which is prized for PC compatibility, Game Salad for OSX, and Construct 2, which is PC-only but has a powerful engine to manage game physics and interactions. These engines all assume a certain type of player interaction: they are designed to enable designers to produce specific types of games, such as a "shooter" or a "platformer," similar in style to the Call of Duty franchise or Nintendo's Mario series. The interactions available are easily understood as a language of action by their players, provided players have previous experience with digital gameplay.


\subsection{Twine}
Twine is a game engine that allows designers to build HTML5 text narratives that branch into a choose-your-own-adventure game. screenPerfect was inspired by the popularity of both Twine and video on the internet. Twine encourages expressive type styling and elements of multimedia, including music, and game screens, but does not require these elements for a complete interactive narrative. Twine did not yet support video narratives in 2013.

The Twine engine was popularized by DIY gaming celebrity Anna Anthropy in her 2012 book Rise of the Videogame Zinesters \parencite{anthropy}. Since then, hundreds of Twines - the adopted term for narratives built in Twine - have seen release.

\subsection{Multiscreen Video Technology}
Dual screen technology, or more accurately, multi-screen synchronous web technology, is one of the big new ideas being heavily backed by Google in 2013. As a consequence, its Chrome browser has been designed to support software developed with a specific suite of frameworks, many of which are wholly supported by Google. 
This being said, Google supports Node and Chrome both, so multi-screen technology using web browsers is accessible to people for no more investment than a new language. screenPerfect relies on Node.JS, which is based on Google's V8 engine.

The architecture of screenPerfect is wholly new, but the concept is based on the Dataton Watchout system, which encourages producers to develop large multi-screen single video experiences on custom hardware. Dataton Watchout costs approximately forty thousand dollars per installation, which makes an inexpensive alternative appealing from a creative standpoint. screenPerfect permits people to use existing hardware to synch multiple videos to one set of controls. This is also distinct from another related tool, ChromeCast, that allows people to wirelessly pair a television with a touchscreen for control and consumption of the touchscreen at a larger size.

\section{Theory and Politics}
Cixous' Laugh of the Medusa predates the computer age, but perfectly and predictably describes the opportunity present in programming - which is a form of writing - within Laugh of the Medusa:
\begin{quote}
"Write, let no one hold you back, let nothing stop you: not man; not the imbecilic capitalist machinery, in which publishing houses are the crafty, obsequious relayers of imperatives handed down by an economy that works against us and off our backs; and not yourself." 
\textit{\parencite{cixous}}
\end{quote}

In this passage, Cixous chides her readers for not giving themselves the permission to write, because writing is reserved for those who might be published. This is similar to game-makers who might not produce, merely because the engines are inaccessible, or distribution unlikely. Women have had a long history in technology.
Ada Lovelace, daughter of Lord Byron, has been identified as the first programmer \parencite{plant}. The ability to put rules in order, to work backwards and forwards from a desired result all along the path of the machines, is a characteristic much sought for in both programmers and game designers. Both roles are responsible for rule systems that will dictate a predictable result.

In a straightforward way, ladies may not possess uncomplicated positions of economic advantage within a patriarchy: to do so would be to "have it all," a famously complex desire which is reanalyzed every year in popular press. The balancing factor is housework and the social expectation of family, which still places a gendered burden on women to produce both children and home. This system of expectations re-creates itself in each new trade as it arrives, in fields as far apart as World War 2 factories \parencite{summerfield}, Victorian mills \parencite{baskerville}, and lately, code factories in Manhattan and San Francisco versus one's own kitchen, canning \parencite{newdomestic}. Computers have quickly become a good job with a good chance to better one's life. It is presently popular to assert that in the future, there will be two types of lives:

\begin{quote}
"...those who tell computers what to do, and those who're told by computers what to do."
\textit{Marc Andreesen, Andreesen Horrowitz}
\end{quote}

This is broadly accurate, but not specifically. Computers are a tool, and the way that the tool is presented, held, and used, dictates the results. A computer - a machine for automating an interaction - can be made simpler or more complex to use. The device may be changed to an amplifier of force, made more opaque, or made clearer for those who choose not to learn to code, but can still understand systems of logic. If this then that is not a complex instruction set. The complex instruction sets should, rather than being encouraged to control people, be developed to be under the control of people.

screenPerfect has been designed to present the idea that a simpler system will result in a more diverse body of work. It dictates nothing whatsoever about content, presenting instead a simple system of switches that permits the author the broadest possible control over simplified interaction sets. It is implied that these interactions will lead to a coherent narrative, but it does not dictate what content an artist might use. The voice of the artist is brought to the forefront of the work, rather than the voice of technology.

By simplifying the process of game design and displacing its nexus from the computer to other design tools, technology is repurposed to be one tool among many. This displaces technology's primary position and refocuses the work on the intent of the artist. This is an implicit system of resistance to the narrative of technology: people can once again tell computers what to do, and extend themselves via their tools.

\section{Cixous, Embodiment, and the Game Jam}

Many of the participants of the game jam discussed in Chapter 4 produced, with the simple yet powerful tools provided, narratives centered on their own embodied or disembodied experience. The only group to resist doing this were the younger OCADu students who came to the jam via a game design practice, rather than a filmmaking, animation, photographic or other new media experience.

Of the games produced and finished, PornGame by Max Lander is directly about the experience of sexuality as it applies to a machine. Grimoire is about the loss of personal control following the finding of a "grimoire" or textbook. Kill Fuck Marry was about bad decisions when it came to dating and sex, OM about a practice of embodied mindfulness, and Cyborg Goddess went straight to Donna Haraway's Cyborg Manifesto in a literal interpretation. Glitch.95, though mainly interested in the glitch aesthetic, is a depiction of the beginnings of an internet relationship. Puppet story is about a genderless creature coming to life, Pinocchio-esque - or perhaps more Galatea, though that is projection.

This is simply an observation, but the most personal and body-centric stories came from this jam, which was nominally about only the use of a tool to express a new format of interaction software. This seems tied to Cixous' assertion that 'You can't talk about \textit{a} female sexuality' and her immediate followup that 'Time and again I ... could burst forth with forms much more beautiful than those which are put up in frames and sold for a stinking fortune' \parencite{cixous}. This is the core of what games critics are discussing when independent games, or new art forms of any type, are being spoken about publicly. Cixous's conception of masturbation and writing as a unified activity practiced in secret can be easily associated to the idea of imposter syndrome, the thought that one could not possibly be "good enough" to break into a creative or high-level industry \cite{imposter}. 

Imposter syndrome is a major problem in the technology industry, and in games even moreso: because code is a high-level creative practice, with a great deal of material reward when done properly, and games are similar, the stakes are high. I have conflated games and code because both are systems of organized rules: game design is a type of code, where one purses conditioned response and engagement through good user experience design. It is easy to back off from both practices or to pursue them exclusively as a hobby, rather than believe in them as a trade. Does one wish to join the commercial order? Is it necessary to be validated, to have commercial success, or is it simply another demand on the work, that the work sell to prove that it is fundamentally worthwhile? These questions are outside the scope of this paper, although my personal opinion is in line with Cixous: whether the money follows or not, a plurality of voices in any creative practice is important for its own sake.

Cixous' throwaway line, of "arid millenial ground to break" seems especially poignant in light of the generational nickname of the jam participants. 