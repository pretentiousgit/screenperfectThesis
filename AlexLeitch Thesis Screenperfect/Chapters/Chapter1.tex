% Chapter 1

\chapter{Introduction: Researching A Better Toolset} % Main chapter title

\label{Chapter1} % For referencing the chapter elsewhere, use \ref{Chapter1} 

\lhead{Chapter 1. \emph{Researching A Better Toolset}} % This is for the header on each page - perhaps a shortened title

%----------------------------------------------------------------------------------------

\section{Research Question}
My aptitude for computer technology combined with my artistic environment has resulted in me repairing a lot of machines. Computers need fixing for many reasons, but usually, they need fixing because the people making software have little experience with people whose primary skills are not related to computer use. The developers and their organizations are isolated from the broader population of people who use these devices. This is problematic, especially in the case of the arts. 

The arts and the humanities are tasked with producing both North America’s culture and its record of its culture. Computers are how we produce this work in 2013, and they are not simple or direct machines. A computer in 2013 is a running compromise between software that works pretty well most of the time and hardware that is generally reliable. To use these tools is a specific skill set. Using them well is as specific as using a paintbrush well. The software developed from that skillset needs to compromise. The number of people for whom computer use will be a delight in and of itself is as limited a group as for any other specialized skill. This is problematic, because art production and artistic project concepts is difficult even without the boundaries raised by software tool challenges. The more limited and specific the skill set required to use contemporary software tools, the more difficult it is to include a diversity of voices in the cultural production of genuinely contemporary work. 

When artists are excluded from technology, culture splits on lines of privilege. There are then artists, who make art, and technologists, who make technology but do not see themselves as particularly creative, or, more to the point, particularly involved in any responsibility for the ideas encoded in their work. Technology is not neutral. It is authored, and where there is authorship, there is a responsibility for ideas. When large groups are left out of communication media, particularly those tasked with producing the language with which culture speaks to itself, there comes a disconnect in our sense of self, and how we are publicly represented. 

Video games are the newest of the cultural production engines. A good game, as described by the MDA, first involves engaging mechanics – the loop systems of reward and scoring – then graphics to create a world, then sound to fill out that world. The controls are weighted this way and that, but the most profitable games - referred to as AAA or triple-A properties - are presently power fantasies. There are other fantasies out there than the ones at the end of a gun.

The value of a broad range of voices in any cultural practice should be self-evident, but video games are presently unique in their high-level uniformity. Unlike any other discipline, the basic tools for games tend to be expensive, and the toolsets that exist to generate games support specific mechanics in similar-seeming worlds. This means that art games are unfairly compared to commercial games produced with the same media. The commercial games set the ground of the conversation, which denies the subjectivity of the producer of the second-stage tools. These producers - developers and game designers - are disappeared into the system, their work only revealed by which elements an artist can or cannot subvert. This is problematic, as the second-stage producer in large part controls what the final creative director can make happen. Without cooperative tools, it is challenging to make new things.

The hardware restrictions of new games are less than advertised. While Moore’s Law is slowing down, reducing the expected increase in brute speed of linear processes in central processing units, input devices are thriving. The new research in games is no longer how to get the best graphics, but rather in how to get the most interesting experience for the investment.

The problem of interesting experiences is not trivial. Video games offer an economically advantageous distraction engine, a way to enact an artificial life. Allowing a diverse range of voices easy access to portray their own games, their own alternate or idealized modes of being, is a way of making those voices more real, of offering an alternate human experience to the “asshole simulator” {\cite Bissell} genres manufactured at much higher budgets.

My research question is to ask how to make some of those tools accessible, for people who do not want to make conventional digital games. I am basing my research on the Twine engine, popular for authoring branching text narratives, and adding an element of irony by including video, to make a video||game. This is an exploration of what a good creative tool can be, how is it different from a bad tool or an inaccessible tool.

My secondary research questions address why is it important to produce polyphony of artistic experience, and how to encourage more diverse voices in this new area of cultural production. How does working collaboratively affect tool development, and what alternate paths to learning are provided by this style of work? What are the underlying assumptions that should be included in code? What ways of working are most reproducible? Is code itself a creative practice, and how can that creative practice be described?

%----------------------------------------------------------------------------------------

\section{Initial Approach}
My initial approach and research method to this work is based on the Agile Manifesto, a software development methodology that opposes siloed, top-down software development. I have paired Agile development with the work of Helene Cixous, whose “Laugh of the Medusa” provided a template for ecriture feminine, an argument for women writing of their own experience in order to be made visible. I have also examined works such as Vera Frenkel's \textit{String Games} and some of the history of conceptual video art within Canada. 

This work is related to various texts of feminist and woman-oriented cybertheory that have appeared in the years since: Haraway’s Cyborg Manifesto, TIQQUN’s Preliminary Materials Towards A Theory of the Young-Girl. All of these are academic constructions of femininity as it is seen in relation to technology: they are feminist in the formal sense of the word. There are other senses of the term, which I will not be examining within the paper. Rather than expressing this work in context with Cixous as \textit{écriture feminine}, I will be using Cixous as a reference for the idea of the alien perspective as a perspective of resistance within a means of expression controlled by a neutral-to-hostile majority perspective. 

This approach addresses women as an alien construct to the more conventional world of technology, which has been recently associated with a masculinist performance that is unnecessary for the pure structure of good rules and the development, through that, of good software. 

My initial approach is to pair with an artist who had a game idea, take that idea, and then make it reproducible. Reproduction is, after all, the province of cyborgs: we control our biological systems, and in doing so, we have conquered what was once a hard-built destiny. By making the consequences of biological sex into a more pliant construct of gender, we have transformed what we must be to what we might be.

This work has been difficult, but I believe it to be important, not so much for the software itself - a proof of concept - but because it is important to provide software that permits people to access new technology. I am not alone in thinking this. The Arduino project, a microcontroller designed to make electronics more accessible to artists, and the Processing project, a simplified version of Java intended to improve the experience of scripting visual effects for artists, are both dedicated to the same ideals. Underlying both projects, as well as the broader Maker movement, is an ideal of participation in one’s own work.

Put simply, people who write code, particularly using the Agile methodology, are engaging in creative practice themselves, and they then display that creative practice through the artists who repurpose their work. This is a different design pattern than technology conventionally pursues, where the work is designed in isolation and released. I am using the Agile method to develop software because it does not require that one knows what the end shape of the software will be in order to pursue the end goal. Agile requires instead that developers pursue goals in sight, always keeping their development loose enough that they can repurpose their work without much effort, and it is ideal for working with artists.

What might be is a developmental model for software that permits transition in scope from the singular, minimum-viable-product model, to a model that builds on itself until complete. A small, perfect thing that does one thing very well, which permits artists to pair their own practice with a software built expressly to make their lives easier, for not too much money. This is important, because my initial approach assumes artists to be undercompensated for their work. Although artists are the central agents of production of all the cultural capital - the invisible value - of the culture industry, they are not the prime beneficiaries of the financial system that backs, stores, and distributes that capital. Therefore, software for artists needs to be inexpensive, and set up to be almost trivially easy to use. This reserves the value of scarcity to the ability of the artist, rather than applying the majority value to the role of the engineer. This kind of invisibility is the invisibility of good management, of any type of good administration. Like housekeeping, code recedes until something goes wrong.

To test this idea, I have approached people to produce video|games with the screenPerfect software in the context of a voluntary game jam – a type of collaborative space where participants work with digital tools to generate new, raw games in a limited window – and then compiling the results into an arcade machine for presentation. 

%----------------------------------------------------------------------------------------

\section{Structure of the Remainder of this Document}

\subsection{Chapters}
The remainder of this document is structured as follows. Chapter 2 covers various methods used in my research to examine the conceptual importance of accessible-technology artist tools, including the Agile Manifesto. Chapter 2 also sets out the restrictions and main theoretical texts that underly my premise of what constitutes an accessible tool. It also addresses a list of tools used to produce the thesis proper, including git revisions, LaTeX, and their shortcomings. 

Chapter 3 addresses theoretical documents, and how they relate to the process of crafting solid software from an artistic/conceptual work perspective. This is where I have placed works by Cixous as well as an examination of Galloway's essays on gaming as algorithmic culture. I will also delimit which texts I consider useful for this work, and how to ground a video||game in the broader context of recent Canadian art history, including the work of Vera Frenkel. 

Chapter 4, Design Research, is the chapter where I work through the process of developing the game engine with reference to the benefits of open, closed, and ideologically-driven software. There is some strategic foresighting here through Doctorow's Pirate Cinema, the text I have used as a How Might We for what a collaborative theatre might be. I have also included reference to the Brechtian active audience, which is useful for gaming. Chapter 4 also contains the bulk of my research on collaborative practice within communities.

Chapter 5, The Trouble With Amateur, addresses some of the questions raised by Galloway in his \textit{Gaming: Essays on a Algorithmic Culture}, which includes an examination of why games built in resistance to gaming tropes are largely unsuccessful and unpopular to play. This chapter is also where I will examine some of the problems of making free content easier to provide to the internet, which includes an economic examination of how the value curve of creative practice goes flat as accessibility increases.

Chapter 6 is my conclusion, a restatement of my arguments, and the source of some optimism.

\subsection{Appendices}
Appendix A is my annotated literature review.

Appendix B is a compilaton of github commit comments over the course of the writing of the game, which detail the direction of how someone reasonably confident with computers and programming learns a new language. 

Appendix C is the code record of screenPerfect proper.

Appendix D is a list of the games made to date with screenPerfect and their installation sites, along with links to where they might be found for future installation.
