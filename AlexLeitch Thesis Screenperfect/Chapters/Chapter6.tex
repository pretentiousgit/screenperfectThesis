% Chapter Template

\chapter{Conclusion} % Main chapter title

\label{Chapter6} % Change X to a consecutive number; for referencing this chapter elsewhere, use \ref{ChapterX}

\lhead{Chapter 6. \emph{Conclusion}} % Change X to a consecutive number; this is for the header on each page - perhaps a shortened title

%----------------------------------------------------------------------------------------
%	SECTION 1
%----------------------------------------------------------------------------------------

\section{Conclusion}
The work that has gone into the production and release of screenPerfect is not inconsiderable. As a creative project based in grounded theory and reflective practice, code is a tricky thing to pin down. It must be declarative, yet it reveals the internal architecture of the people who write it. To write code is to be a craftsperson, and to be a craftsperson is to reveal some of yourself with every line.
Programming solo is as rewarding as any other solitary occupation, yet it leaves many loose ends. An excellent piece of software is likely to require input from a wide array of specialists in graphic design, interface development, and logic. There is an inevitability to induced flaws - bugs - that cause the program to fail. Once complete, it is likely that finished software will fall out of fashion: software is the unfinished symphony of the 21st century. Just as there is no way to call a piece of writing finished, because another word can always be added or cut loose, code is subject to scope creep. Code changes. It is not a silkscreen, once pulled and forever finished. It is not a painting, which, dried and delivered, is safe until the conservators come for it. Code lives, like writing, in context and within an ecosystem.
Unlike writing, code answers to its context; without the machines to whom it speaks, it is without consequence. Within those machines, it may have a concrete effect on the world around it, and for that reason it continues to be valued; this is the craftsmanship that, unlike art, continues to make a living. Code cannot be set aside; although it will work as intended, it will break without permission. To code is to attempt to write a coherent world into being, to attempt to factor in all the diverse chances within.
In Cixous’ Laugh of the Medusa\cite{medusa}, she expresses that to be considered real, women must write for themselves. In my thesis, I have extended this to the world of code: one must write after one’s own interests, because to take only that work which is assigned is to fall behind. Writing a projection/presentation/game system has been work designed to address the problem of what, exactly, a game is, or what is a valuable piece of work. screenPerfect is designed to evaporate, leaving only the experience of its content to represent itself. It does not care what your content is, or to whom you’re serving it, or where. The emphasis is on allowing your audience to experience things as quickly and easily as possible.
Works produced using screenPerfect can be displayed anywhere in the privileged world; anywhere a series of screens and a single server can be set up. This emphasis on experience moves the interaction sphere into the world. The display of video-art and collaborative gameplay made possible through screenPerfect can be anywhere at all, and indeed works best at night, outdoors, in temporary installations. These are the new/old/new exhibits, the unmissable one-time-only parties, the experience that happens in a hard to access place but leaves no marks for future visitors to interpret.

The reflective portion of this research has been to address the question of use and pragmatism, as well as what constitutes research within an artistic context. The answer is difficult to quantify; research is pursuing a read, book-learning, critical examination of a practice, as all art is practice. Art is fundamentally blue collar, as is coding; both are works of craftsmanship, both can be helped along by automation, but ultimately, their declaration within a finished state is dependent upon the person who has produced them. The blue-collar trade of art has been sold out by the academy, linked tightly to the idea that thinking about a thing is more valuable than working on a thing oneself. Coders are legendarily difficult to organize, resistant to unions or to the thought that they are themselves labourers.

The newest video games try to dream worlds past being human, and most fall far short. Rather than permitting a wide exploration of possibilities, many possibilities narrow to the point of a gun, to the same forearms for twenty years (twitter). At the heart of screenPerfect is the idea that we can pull away from artificial distance and have instead on-site participation, unique experiences that project real, contemporary art into real, contemporary spaces. We can have events anywhere, and these events can bring anyone together, with even terrible technology. Underlying the architecture of this code is the idea that all the different flavours of classism should be undone with the opportunity to see amazing things, no matter who you are. Space should belong to the people who occupy it most often, not only the people who pay for it at a distance.

Learning new programming languages is always a challenge, as is enacting a pragmatic device from the perspective of art theory. There is the concern that these works are not for anything, not for a job or an application or a visible piece of content, a series of products released to do something in the world. Many applications, games and devices are released into the world that simply consume resources for the sake of their own consumption, simply to show that the people involved have the resources to spend. A new digital toy is frequently out of reach for the vast group of people who depend on their existing technology to work for as long as they can make it do so, and therefore, these tools are designed in the same way as the first university mainframes: they live somewhere else, and can be accessed by even the most unfortunate smartphones, five years out of date and slow. The idea is to permit this sort of advanced media to get to places that I do not expect it to turn up, in places that are specifically not shiny. The idea is to use existing technology to permit exploration: use what there is, and then make it greater, rather than interject an external device that will require people to spend resources to use.

Ultimately, this sort of software development is about permission. Permission for people to do what they like in the spaces they need to occupy and use, including the digital space. This is not about developing a single app or a platform that will be easily marketed, but is instead about focusing on the value of the exclusive experience. The Game Jam, which is a located, people-in-the-room bit of enthusiasm, and the value of a small community which helps one another to develop, is reflected in the variety of games produced for the system. This is about exploring the possibilities of a space designed to share an experience through a personal connection.

This part is about the different values permitted to different classes of entertainment. At  a high level, art can afford to be alienating, and indeed is frequently valued more highly for its power of alienation than for any other thing. This is a distinction made especially true in video games, where the power of “fun” tends to be valued highly for its commercial properties. If a game is not “fun,” all elements of its interactive powers of storytelling cease. This means that games which are not “fun” are widely called by other names - interactive new media art, for example. 

screenPerfect is designed to permit the development of games that are not only not necessarily fun, but which may be narratively incomplete, or nonsensical, while still being absorbing. These are games that do not require reading, but permit themselves to be experienced as deeply as a given group of readers wish to experience them. The tool can be perverted: perverse use is built in, with video copyright being at such a premium. It can be used to host experiences in galleries or in warehouses, by people with minimal technical knowledge and little ability to mask their normal online activities. This is a device to let people make maximum use of the devices they already have, to host a dance party on a subway or a massive art tour through a gallery. The demonstration content may be upsetting, but the access permitted is broad. This is about sharing things, for the better.