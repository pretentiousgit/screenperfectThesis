% Chapter Template

\chapter{Conclusion} % Main chapter title

\label{Chapter6} % Change X to a consecutive number; for referencing this chapter elsewhere, use \ref{ChapterX}

\lhead{Chapter 6. \emph{Conclusion}} % Change X to a consecutive number; this is for the header on each page - perhaps a shortened title

%----------------------------------------------------------------------------------------
%	SECTION 1
%----------------------------------------------------------------------------------------

\section{Conclusion}
The work that has gone into the production and release of screenPerfect is not inconsiderable. As a creative project based in grounded theory and reflective practice, code is a tricky thing to pin down; it must be declarative, yet reveals the internal architecture of the people who write it. To write code is to be a craftsperson, and to be a craftsperson is to reveal some of yourself with every stroke of the plane. 
The tools themselves, though, are a concern. Programming solo is as rewarding as any other solitary occupation, yet it leaves many loose ends. An excellent piece of software is likely to require input from a wide array of specialists in graphic design, interface development, and logic. There is an inevitability to induced flaws - bugs - that cause the program to fail. Once complete, it is likely that finished software will fall out of fashion; the unfinished symphony of the 21st century. There is no way to call a piece of writing finished, because another word can always be added or cut loose - the best that can be managed is a date of publication. Similarly, code is subject to scope creep. It is not a silkscreen, once pulled and forever finished. It is not a painting, which, dried and delivered, is safe until the conservators come for it. Code lives, like writing, in context and within an ecosystem.
Unlike writing, code answers to its context; without the machines to whom it speaks, it is without consequence. Within those machines, it may have a concrete effect on the world around it, and for that reason it continues to be valued; this is the craftsmanship that, unlike art, continues to make a living. Code cannot be set aside; although it will work as intended, it will break without permission. To code is to attempt to write a coherent world into being, to attempt to factor in all the diverse chances within.
In Cixous’ Laugh of the Medusa, she expresses that to be considered real, women must write for themselves. In my thesis, I have extended this to the world of code; one must write after one’s own interests, because to take only that work which is assigned is to fall behind. Writing a projection/presentation/game system that emphasizes privacy has been work designed to address the problem of what, exactly, a game is, or what is a valuable piece of work. Most systems are designed to be public, to emphasize content or possibly advertisements, to be clear. Though functional, screenPerfect is not designed to do any of these things. Instead, this is software that disappears. It does not care what your content is, or to whom you’re serving it, or where. The emphasis is on allowing your audience to experience things as quickly and easily as possible.
Works produced using screenPerfect can be displayed anywhere in the privileged world; anywhere a series of android/iDevices and a single server can be set up, to create a collaborative experience. This is designed to serve video first, and then to move through static content, sound, any experience in almost any context.
This emphasis on experiences that are not solo, or hidden away inside a computer lab or a movie theatre, moves the interaction sphere firmly back to the real world. The display of the video-art and collaborative gameplay made possible through screenPerfect can be anywhere at all, and indeed works best at night, outdoors, in temporary installations. These are the new/old/new exhibits, the unmissable one-time-only parties, the experience that happens in a hard to access place but leaves no marks for future visitors to interpret. Collaborative spaces are difficult to build, and mass streaming technology is impossible to privatize; therefore, screenPerfect allows the possibility of being entirely removed from the cloud; no-one can see these videos except the people present at the time, and no-one can quite comprehend the overwhelming inputs outside the event. 
The practical uses for a one-time event control with local servers are fairly obvious. The first application is for large-scale events, such as Alternate Reality Games, which may need a totally private communication channel for players, which would be difficult to pirate or resell. The next is to move those ARGs to spaces like Yonge-Dundas square, where monitors could be tuned to a local-only website to extend the reach of the experience, which is in real time. By providing such spaces to interact with, an artist can redefine the scale of their video work; it no longer lives in a laptop or a theatre, but outside, and controllable from within the crowd at the base of the display.
If this were an open connection, such a permission might prove challenging; access to the open internet invariably brings up privacy concerns. Therefore, screenPerfect is limited in context. It will allow access to the screenPerfect presentation application, and what that application can display is pre-set by the artist. Therefore, although a crowd of dozens could control what is going on at any given time, only a pre-set group of experiences (videos, still objects) can be shared. This saves on public embarrassment, and minimizes the staff required to support any given screenPerfect installation. 
The reflective portion of this research has been to address the question of use and pragmatism, as well as what constitutes research within an artistic context. The answer is difficult to quantify; research is pursuing a read, book-learning, critical examination of a practice, as all art is practice. Art is fundamentally blue collar, as is coding; both are works of craftsmanship, both can be helped along by automation, but ultimately, their declaration within a finished state is dependent upon the person who has produced them. The blue-collar trade of art has been sold out by the academy, linked tightly to the idea that thinking about a thing is more valuable than working on a thing oneself. Coders are legendarily difficult to organize, resistant to unions or to the thought that they are themselves labourers.[This part is no good, things start to go really haywire in here.

Alex Leitch, 2013-09-13 3:39 PM]
The design of video games is a renewal of old ideals of propaganda and imagination as previously permitted almost exclusively by theatre. Rather than being restricted to[There is a maaaajor topical break here between art and gaming.

Alex Leitch, 2013-09-13 3:40 PM] watching, an audience can now be, and do, their own experience. The experience is not unmediated, but is a collective participation. This makes some elements more difficult, and others simpler. We can make a world, and make that world better, but the world we make will still be influenced by the things we ourselves have experienced.[How do we make that world? 

Alex Leitch, 2013-09-13 3:40 PM] Even the clearest software architecture is still an architecture that must adhere to the mechanics of being human.
The newest video games try to dream worlds past being human, and most fall far short. Rather than permitting a wide exploration of possibilities, many possibilities narrow to the point of a gun, to the same forearms for twenty years (twitter). At the heart of screenPerfect is the idea that we can pull away from artificial distance and have instead on-site participation, unique experiences that project real, contemporary art into real, contemporary spaces. We can have events anywhere, and these events can bring anyone together, with even terrible technology. Underlying the architecture of this code is the idea that all the different flavours of classism should be undone with the opportunity to see amazing things, no matter who you are. Space should belong to the people who occupy it most often, not only the people who pay for it at a distance.
[Repetitive.

Alex Leitch, 2013-09-13 3:41 PM]Having produced this work, which is fundamentally a software set designed for parties where people are too nervous to interact, has been a tremendous challenge. Learning new programming languages is always a challenge, as is enacting a pragmatic device from the perspective of art theory. There is the concern that these works are not for anything, not for a job or an application or a visible piece of content, a series of products released to do something in the world. Many applications, games and devices are released into the world that simply consume resources for the sake of their own consumption, simply to show that the people involved have the resources to spend. A new digital toy is frequently out of reach for the vast group of people who depend on their existing technology to work for as long as they can make it do so, and therefore, these tools are designed in the same way as the first university mainframes; they live somewhere else, and can be accessed by even the most unfortunate smartphones, five years out of date and slow. The idea is to permit this sort of advanced media to get to places that I do not expect it to turn up, in places that are specifically not shiny. The idea is to use existing technology to permit exploration, and inversion of the One Laptop Per Child project; use what there is, and then make it greater, rather than interject an external device that will require people to spend resources to use.
[Chapter needed on OLPC project and how it relates to the dream of accessible technology, some background from Negroponte and the MIT-classist dream that kids should have this before clean water.

Alex Leitch, 2013-09-13 3:41 PM]Ultimately, this sort of software development is about permission. Permission for people to do what they like in the spaces they need to occupy and use, including the digital space. This is not about developing a single app or a platform that will be easily marketed, but is instead about focusing on the value of the exclusive, the small, the private, the well-built-for-humans space; these spaces are difficult to construct effectively and require more maintenance than the average Toronto garden.[Colloquial tone. Journalistic.

Alex Leitch, 2013-09-13 3:42 PM] This is about exploring the possibilities of a space designed to share an experience through a personal connection.
The content of the work itself is designed to unsettle. As an artist, I have long cared about the powers of horror[Chapter contrasting the power of body-horror with the dream of a technological cleanliness that is promoted and misunderstood by the OLPC project; these are concerns that are classist in the way that the genocide of the people who cannot access healthcare is. 

It would be interesting to bring Roiphe and the art of the white lady writing into this. As a white lady writing, anyway.

Alex Leitch, 2013-09-13 3:42 PM], the power of things being slightly not-right to make people deeply, passionately uncomfortable. Therefore, the content of the work and the final setup is designed to confuse. It is built to be difficult to experience all at once, without others with whom to cooperate. This part is less about joy than it is about the different values permitted to different classes of entertainment. At  a high level, art can afford to be alienating, and indeed is frequently valued more highly for its power of alienation than for any other thing. This is a distinction made especially true in video games, where the power of “fun” tends to be valued highly for its commercial properties; if a game is not “fun,” all elements of its interactive powers of storytelling cease. This means that games which are not “fun” are widely called by other names; interactive new media art, for example. 
screenPerfect is designed to permit the development of games that are not only not necessarily fun, but which may be narratively incomplete, or nonsensical, while still being absorbing. These are games that do not require reading, but permit themselves to be experienced as deeply as a given group of readers wish to experience them. The tool can be perverted; perverse use is built in, with video copyright being at such a premium. It can be used to host experiences in galleries or in warehouses, by people with minimal technical knowledge and little ability to mask their normal online activities. This is a device to let people make maximum use of the devices they already have, to host a dance party on a subway or a massive art tour through a gallery. The demonstration content may be upsetting, but the access permitted is broad. This is about sharing things, for the better.