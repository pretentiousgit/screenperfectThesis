% Chapter Template

\chapter{State of the Art} % Main chapter title
\chapter{State of the Art} % Main chapter title

\label{Chapter3} % Change X to a consecutive number; for referencing this chapter elsewhere, use \ref{ChapterX}

\lhead{Chapter 3. \emph{State of the Art}} % Change X to a consecutive number; this is for the header on each page - perhaps a shortened title

%----------------------------------------------------------------------------------------
% Here you review the state of the art/design/science/research relevant to your thesis. The idea is to 
% present the major ideas in the state of the art right up to, but not including, your own research and contribution (critical analysis comes a little bit later). You organize this section around ideas, 
% and not by author or by publication.
%----------------------------------------------------------------------------------------

%----------------------------------------------------------------------------------------
%	TOOLS
%----------------------------------------------------------------------------------------

\section{Twine and Accessible Tools}
\begin{itemize}
\item[\tiny{$\blacksquare$}] Twines are super accessible but still fettered by presentation
\item[\tiny{$\blacksquare$}] CYOA is a basic interaction
\item[\tiny{$\blacksquare$}] This is often described as "not a game"
\end{itemize}
\subsection{Unity}


%-----------------------------------
%	Game theory - Galloway
%-----------------------------------

\section{Game Theory: Galloway, Anna Anthropy}
\begin{itemize}
\item[\tiny{$\blacksquare$}] Games should be personal
\item[\tiny{$\blacksquare$}] Chapter on artistic mods
\item[\tiny{$\blacksquare$}] Mention Merrit Kopas' work on LIM and HUGPUNX and etc.
\end{itemize}

%-----------------------------------
%	Cyborg Feminism current thinking
%-----------------------------------
\section{Cyborg Technology: Haraway}
\begin{itemize}
\item[\tiny{$\blacksquare$}] Emma's paper at DiGRA
\end{itemize}

\section{What About All The Gays And The Internet Sex}
\subsection{Cut this, even though it's clearly pretty relevant and f*cking everyone just turned out porn or violence games.}
In a more complex circumstance, however, the elimination of women from the winning populace of the creative computer lexicon has an uncomfortable resonance with the minimization of the careers of female creatives everywhere. The Guerilla Girl's work through the late 1980s and the 1990s to today expresses the gender gap nicely: ladies just don't get featured for anything in art magazines except messy bedrooms (\cite{emin}). This sexualization and obsession with the female as an object was extremely popular during the late 1990s, eventually resulting in Sex in the City, a counter-argument to Riot Grrl if ever there was one. 

This is addressed most fiercely by the French anonymous collective TIQQUN in their \textit{Preliminary Materials Toward a Theory of the Young-Girl} (\cite{tiqqun}), released in 1999, just as the internet was really getting started. The Viridian manifesto by Sterling was released in the same year. All of these materials point towards a terror of the perfect image torrent that the internet unleashed, particularly the pornography made suddenly available on demand at any time, turned from a rare shame to a public utility pumped into one's house, like water or electricity, overnight.

The thing is, messy bedrooms hold attention like clean ones never will. The cult of youth carries on, because youth is when the majority population are still able to ask questions, and when they still have both disposable income, and time (\cite{economyterrible}). The hardest-core gamers - and doesn't that description just have too much in common with pornography already - are stereotyped as single males between the ages of 18 and 35, although they skew older. Mainstream games consoles are dedicated to murdering things over and over and over again, in different varieties of the same acts, which bear a strong resemblance to mainstream pornography. At the same time, this depiction is complicated by the stereotype of a gamer as a male someone who does not, to put it indelicately, get laid very often.

Unfortunately, nothing whatsoever about the stereotype is true. While major gaming systems put on increasingly masculinized trade shows designed to turn out remarkably similar entertainment properties - the triple-A games that bring in hundreds of millions of dollars (\cite{valueofaaa}) - the queers are hiding out elsewhere, and they are taking the art with them. Anna Anthropy's crew of trans indie game-makers are shaking things up by producing low-budget things which take advantage of easy to use tools to produce games that express a personal narrative, and it is this tradition that screenPerfect is intended to work with.