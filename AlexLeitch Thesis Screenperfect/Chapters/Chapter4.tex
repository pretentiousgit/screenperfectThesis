% Chapter Template

\chapter{Industry Engagement and Design Research: Community Collaboration in Software Development} % Main chapter title

\label{Chapter4} % Change X to a consecutive number; for referencing this chapter elsewhere, use \ref{ChapterX}

\lhead{Chapter 4. \emph{Industry Engagement and Design Research}} % Change X to a consecutive number; this is for the header on each page - perhaps a shortened title

%----------------------------------------------------------------------------------------
%	SECTION 1
%----------------------------------------------------------------------------------------
\begin{itemize}
\section{Industry Engagement}

\item[\tiny{$\blacksquare$}] Describe Miso's involvement here

%-----------------------------------
%	Dames Making Games
%-----------------------------------
\subsection{DMG}

\item[\tiny{$\blacksquare$}] Describe Dames Making Games here, history of jams

%-----------------------------------
%	What is game jam
%-----------------------------------

\subsection{What is a game jam?}

\item[\tiny{$\blacksquare$}] Explain the Game Jam practice and premise
\item[\tiny{$\blacksquare$}] Ground with references to design charettes.
\item[\tiny{$\blacksquare$}] refer to appendices for jam participant experiences

%----------------------------------------------------------------------------------------
%	Design Research
%----------------------------------------------------------------------------------------
\section{Design Research}

\subsection{Artist Collaboration}

Beginning development from a first position within the arts is unusual, even for an Agile workflow, but in the case of this software, it has been wholly driven by artistic collaboration. I firmly believe that simple tools to relieve the friction points of the artistic process will lead to better art which is more widely available. This is to say, although the developer is themselves a creative who will decide \textit{how} to solve problems, the problems to solve may be better handled by an outside party. This is down to an issue of demand. 

In order for software to exist, and to be seen to exist, it requires an interaction. Unlike a hammer, which takes up space on a shelf, software is essentially a text document until it is used. As \citeauthor{galloway}says, a game is defined by action \citeyear{galloway}. \cite{galloway}


\item[\tiny{$\blacksquare$}] Explain initial process with Hannah's YouTube idea

\subsection{Summer installation of psXXYborg}
\item[\tiny{$\blacksquare$}] Describe installation of psXXYborg.
\item[\tiny{$\blacksquare$}] Include photographs of original van in figures list.

\subsection{Fiction Input}
\item[\tiny{$\blacksquare$}] Describe Cory Doctorow's Pirate Cinema scene. Out-take.
\item[\tiny{$\blacksquare$}] Include photographs of original van

\subsection{GitHub and learning on the internet}
\item[\tiny{$\blacksquare$}] Explain gitHub
\item[\tiny{$\blacksquare$}] Explain open source software, benefits and questions therein.
\item[\tiny{$\blacksquare$}] Explain closed-source software.

\subsection{Problems in Artistic Collaboration}
\item[\tiny{$\blacksquare$}] Artist selling off installation
\item[\tiny{$\blacksquare$}] No ownership for technician
\end{itemize}