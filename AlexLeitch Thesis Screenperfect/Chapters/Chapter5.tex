
% Chapter Template
% Chapter Template
\setstretch{1}
\chapter{Conclusion}\thispagestyle{empty} % Main chapter title

\label{Chapter5} % Change X to a consecutive number; for referencing this chapter elsewhere, use \ref{ChapterX}

\lhead{Chapter 5. \emph{Conclusion}} % Change X to a consecutive number; this is for the header on each page - perhaps a shortened title
\setstretch{2}
%----------------------------------------------------------------------------------------
%	SECTION 1
%----------------------------------------------------------------------------------------

\section{Conclusion}

Artist-led research results in an exploration of not only what is considered popularly reasonable but also what is possible. By anchoring that practice to community feedback from a group of invested users, we can expand software from beyond its original audience to a new section of the population and then work through the specific display and installation requirements for that group. By making the software with a specific user in mind, we both ensure demand and reward involvement with the work; by committing to that work as "real," we can produce new directions for design. Demand here is constructed not as capitalist \textit{massive} or unlimited demand but instead the niche requirements indicated by the new, broader, connected public space. It is possible to have a small overall demand be, in aggregate, worthwhile.

screenPerfect has already been forked into new tools to work towards the likelihood that gamemakers will pursue new video works. Bento Box-Miso's iV fork, which discards dual-screens but encourages branched FMV narratives to be pursued within DMG, was launched at Feb Fatale 2 in early 2014 and has seen some popular uptake. This is an important marker of the overall success of this embedded research. A possible direction for this software is to package it so that games built with it can be installed both locally and remote at the same time in public contexts. The installation of those games in public contexts will be supported by further work on hardware systems for easy, portable, open hardware, that can then be reliably exhibited in new contexts. Distribution of the completed work can take place over the internet as a pre-configured operating system, which, similar to a game cartridge, means that users and gallery specialists need only do a minimum of support work to install, repair, replace and update digital works produced with this system. In addition, this software can be stored on SD cards with backup Raspberry Pi, ready for installation for an indeterminate period of time into the future.

In Cixous' "Laugh of the Medusa" \parencite{cixous} she states that to be considered real, women must write for themselves. This document extends the notion of the written self to the world of code, and through it to computers and the contemporary technological landscape: one must write after one's own interests in order to represent a possibility for what one believes can be real. I have done this by writing a tool that permits an array of interested parties to produce games without a reliance on conventional scripting or game assets. I have then extended screenPerfect with hardware to support a straightforward installation path using materials from a not-for-profit foundation and open-source software, rather than a limited system reliant on strictly commercial ties.

This prototype emphasises allowing an audience to experience digital works as easily as possible in a context controlled by the artist. Works produced using screenPerfect can be displayed anywhere a series of screens and a single server can be set up. This emphasis on experience moves the interaction sphere of art and gaming into the world. screenPerfect's impact is most felt at night, outdoors, or in temporary installations. However, using screenPerfect, it is possible to display video-art and collaborative gameplay anywhere at all. These are the new/old/new exhibits, the one-time-only parties, the experience that happens in a hard to access place yet leaves no marks for future visitors to interpret, such as Nuit Blanche in Toronto, or Burning Man in Nevada. At the heart of screenPerfect is the idea that we can pull away from artificial distance and have instead on-site participation, unique experiences that project real, contemporary art into real, contemporary spaces.

Throughout the development of screenPerfect, I asked a series of questions about capital, monetary and otherwise, creative practice, and preservation in the digital context. This is an effort to directly address the debate about games as art by circumventing it: as we can see by visiting Cixous, any form of creative practice can be dismissed by the dominant voice. A debate about whether a given creative practice is worthy of the name "art" is in fact a form of access control. By testing a piece of software developed through one artist's working practices with a broader audience, we can add resilience to the code. This helps to ensure that it serves both the intended process and supports further experiments. This permits broad access to expression on behalf of new voices while simultaneously stress-testing the work itself. By giving people open tools, we give them the ability to express themselves: if what they would like to say is then something personal, it is my view that there is value in that.

We live within a system of value and valuation. To overcome all elements of that system from within is unlikely - an assertion supported by mathematicians like Gödel as much as feminists like Audrey Lorde - but we do not need to overcome the system to be able to circumvent it. We can choose how much to engage with the system, on what terms. Within that effort we may find answers to specific problems.
