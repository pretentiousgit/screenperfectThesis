
% Chapter Template
% Chapter Template
\setstretch{1}
\chapter{Conclusion}\thispagestyle{empty} % Main chapter title

\label{Chapter5} % Change X to a consecutive number; for referencing this chapter elsewhere, use \ref{ChapterX}

\lhead{Chapter 5. \emph{Conclusion}} % Change X to a consecutive number; this is for the header on each page - perhaps a shortened title
\setstretch{2}
%----------------------------------------------------------------------------------------
%	SECTION 1
%----------------------------------------------------------------------------------------

\section{Conclusion}The work that has gone into the production and release of screenPerfect is not inconsiderable. As a creative project based in grounded theory and reflective practice, code is a tricky thing to pin down. It must be declarative, yet it reveals the internal architecture of its authors. Programming leaves loose ends. An excellent piece of software is likely to require input from a wide array of specialists in graphic design, interface development, and logic. There is inevitability to induced flaws - bugs - that cause the program to fail. Once complete, it is likely that finished software will fall out of fashion. Just as there is no way to call a piece of writing finished, because another word can always be added or cut loose, code is subject to scope creep. 

Code lives, like writing, in context and within an ecosystem. As a form of writing, code answers to its context. Without the machines that run it, code is without consequence. Within those machines, it may have a concrete effect on the world around it, and for that reason it continues to be valued. To code is to attempt to write a way of addressing the world, a single way that must take into consideration all the assumptions of people who tried to address the world before, and, with future-proofing, the world after. 

In Cixous' Laugh of the Medusa \parencite{cixous} she states that to be considered real, women must write for themselves. In my thesis, I have extended this to the world of code: one must write after one's own interests, represent a possibility for what one believes can be real.

Writing a projection/presentation/game system has been work designed to address the problem of what, exactly, a game is, or what is a valuable piece of work. screenPerfect is designed to evaporate, leaving only the experience of its content to represent itself. The system does not judge content. It does not care where you serve your information, or to whom. The emphasis is on allowing an audience to experience things as quickly and easily as possible. Works produced using screenPerfect can be displayed anywhere a series of screens and a single server can be set up. This emphasis on experience moves the interaction sphere of art and gaming into the world. screenPerfect's impact is most felt at night, outdoors, or in temporary installations. The display of video-art and collaborative gameplay made possible through screenPerfect can be anywhere at all, and indeed works best at night, outdoors, in temporary installations. These are the new/old/new exhibits, the one-time-only parties, the experience that happens in a hard to access place but leaves no marks for future visitors to interpret.

The reflective portion of this research has been to address the question of what constitutes accessibility.The newest video games try to dream worlds past human, and most fall far short. Rather than permitting a wide exploration of possibilities, many possibilities narrow to the point of a gun. At the heart of screenPerfect is the idea that we can pull away from artificial distance and have instead on-site participation, unique experiences that project real, contemporary art into real, contemporary spaces. We can have events anywhere, and these events can bring participants together. Underlying the architecture of this code is the idea that space should belong to the people who occupy it most often, not only the people who pay for it at a distance.

Learning new programming languages is always a challenge, as is enacting a pragmatic device from the perspective of art theory. There is the concern that these works are not \textit{for} anything, not for a job or an application or a visible piece of content, a series of products released to do something in the world. A new digital toy is frequently out of reach for the vast group of people who depend on their existing technology to work for as long as they can make it do so, and therefore, these tools are designed to be resilient and basic. The idea is to permit this sort of advanced media to get to unexpected places, in places that are, specifically, not shiny. 

Ultimately, this sort of software development is about permission. Permission for people to do what they like in the spaces they need to occupy and use, including the digital space. The game jam, which is a locative, people-in-the-room phenomena, and the value of a small community which helps one another to develop, is reflected in the variety of games produced for the system. This is about exploring the possibilities of a space designed to share an experience through a personal connection.

There are different values permitted to different classes of entertainment. At a high level, art can afford to be alienating, and indeed is frequently valued more highly for its power of alienation than for any other thing. This is a distinction made especially true in video games, where the power of "fun" tends to be valued highly for its commercial properties. If a game is not "fun," all elements of its interactive powers of storytelling cease. This means that games which are not "fun" are widely called by other names - interactive new media art, for example.

screenPerfect is designed to permit the development of experiences - games - that are not only not necessarily "fun,"" but which may be narratively incomplete, or nonsensical, while still being absorbing. The tool can be perverted: perverse use is built in, with video copyright being at such a premium. It can be used to host experiences in galleries or in warehouses, by people with minimal technical knowledge and little ability to mask their normal online activities. This is a device to let people make maximum use of the devices they already have, to host a dance party on a subway or a massive art tour through a gallery. The demonstration content may be upsetting, but the access permitted is broad. This is about sharing things, for the better.

